\documentclass{article}
\usepackage[utf8]{inputenc}
\usepackage[T1]{fontenc}
\usepackage[english]{babel}
\setlength{\parindent}{0pt}
\usepackage{hyperref}
\hypersetup{
    colorlinks=true,
    linkcolor=blue,
    filecolor=magenta,      
    urlcolor=cyan}
\usepackage{graphicx}
\graphicspath{ {./pic/} }
\usepackage{multicol}
\usepackage{lscape}

\usepackage{fourier,amssymb,microtype,amsmath,gensymb}
\newcommand{\R}{\mathbb{R}}
\usepackage{mdframed,caption,xcolor}
\usepackage{tikz,tkz-euclide}

\title{Seminar 11. Adverse selection and moral hazard}
\author{Xiaoguang Ling \\  \href{xiaoguang.ling@econ.uio.no}{xiaoguang.ling@econ.uio.no}}
\date{\today}

\begin{document}

\maketitle

%%%%%%%%%%%%%%%%%%%%%%%%%%%%%%%%%%%%%%%%%%%%%%%%%%%%%%%%%%%%%%%%%%%%%%%%%%%%%%%%%%%%%%%%%%%%%%

\section{Jehle \& Reny pp.422, exercise 8.7 - Asymmetric information}

(Akerlof) Consider the following market for used cars. There are many sellers of used cars. Each
seller has exactly one used car to sell and is characterised by the quality of the used car he wishes to
sell. Let $\theta \in [0, 1]$ index the quality of a used car and assume that $\theta$ is \textbf{uniformly distributed on $[0, 1]$}.
If a seller of type $\theta$ sells his car (of quality $\theta$) for a price of $p$, his utility is $u_s(p, \theta)$. If he does not sell
his car, then his utility is 0. Buyers of used cars receive utility $\theta - p$ if they buy a car of quality $\theta$ at
price p and receive utility 0 if they do not purchase a car. There is asymmetric information regarding
the quality of used cars. Sellers know the quality of the car they are selling, but buyers do not know
its quality. Assume that there are not enough cars to supply all potential buyers.


\subsection*{(a) Argue that in a competitive equilibrium under asymmetric information, we must have
$E(\theta | p) = p$}


See section on ``Asymmetric Information and Adverse Selection'', pp.~382--385.


\subsection*{(b) Show that if $u_s(p, \theta) = p - \theta/2$, then every $p \in (0, 1/2]$ is an equilibrium price.}

\begin{mdframed}[backgroundcolor=blue!20,linecolor=white]

Recall: for a random variable $\theta \in [a,b] \subset \R_1$, $F(x) = Pr(\theta\le x)$ is called cummulative distribution function (CDF). $f(x)=F'(x)$ is the probability density function (PDF) of $\theta$. We have:

$$E(\theta) = \int^b_a x dF(x)$$

Similarly, for a number $m \in (a,b)$

$$E(\theta|\theta \le m) = \int^m_a x d F(x|\theta \le m)$$

Where $F(x|\theta \le m) = Pr(\theta \le x | \theta \le m)$. 

\medskip 

Obviously, when $x \in [m,b]$, $Pr(\theta \le x | \theta \le m)=1$; when $x \in [a,m]$, according to Bayes' rule,


 $$Pr(\theta \le x | \theta \le m)= \frac{Pr(\theta \le m | \theta \le x) Pr(\theta \le x)}{Pr(\theta \le m)} = \frac{1 \times Pr(F(x)}{F(m)}$$

Therefore,

\begin{equation}
F(x|\theta \le m)=
    \begin{cases}
\frac{F(x)}{F(m)} \quad if \quad x \in [a,m] \\
\ \ 1 \quad \quad if \quad x \in [m,b] 
    \end{cases}
\nonumber
\end{equation}

Thus,

$$E(\theta|\theta \le m) = \int^m_a x d F(x|\theta \le m)=\int^m_a x d \frac{F(x)}{F(m)}$$
\end{mdframed}

Let $p \in (0,\tfrac12]$. Sellers with quality $\theta$ satisfying $p - \tfrac{\theta}2 \geq 0$ sell their cars, or equivalently:$$\theta \leq 2p \, .$$

\medskip

Sine $\theta$ is uniformed distributed in $[0,1]$, the CDF of $\theta$ is $F(x) = x$

\medskip

Then:

\begin{align*}
E(\theta | p) = E(\theta | \theta \leq 2p) &= \int_{0}^{2p}x\frac{dF(x)}{F(2p)} \\
&= \int_{0}^{2p} x\frac{dx}{2p} \\
&= \frac{\left[ \tfrac12 x^2 \right]^{2p}_0}{2p} \\
&= \frac{\tfrac12 (2p)^2 - \tfrac12 (0)^2}{2p} \\
&= p
\end{align*}

So $E(\theta | p) = p$ is satisfied for each $p \in (0,\tfrac12]$.

\subsection*{(c) Find the equilibrium price when $u_s(p, \theta) = p - \sqrt{\theta}$. Describe the equilibrium in words. In
particular, which cars are traded in equilibrium?}

Consider any $p \in [0,1]$. Sellers with quality $\theta$ satisfying $p - \theta^\frac12 \geq 0$ sell their car, or equivalently:$$\theta \leq p^2 \, .$$

Then:

\begin{align*}
E(\theta | p) = E(\theta | \theta \leq p^2) &= \frac{\int_{0}^{p^2}x dF(x)}{F(p^2)} \\
&= \frac{\left[ \tfrac12 \theta^2 \right]^{p^2}_0}{p^2} \\
&= \tfrac12 p^2
\end{align*}

According to the equilibrium result we found in part (a), i.e.
$E(\theta | p) = p$, in an equilibrium we must have:

$$\tfrac12 p^2 = p \Rightarrow (p-2)p = 0 \Rightarrow p^* = 0$$ 

We also have $\theta \leq p^2 \Rightarrow \theta^* = 0$

\medskip

Only $\theta = 0$ will be traded at $p=0$. The market breaks down completely.

\subsection*{(d) Find an equilibrium price when $u_s(p, \theta) = p - \theta^3$. How many equilibria are there in this case?  }

Consider any $p \in [0,1]$. Sellers with quality $\theta$ satisfying $p - \theta^3 \geq 0$ sell their car, or equivalently:$$\theta \leq p^{\tfrac13} \, .$$
Then:
$$E(\theta | p) = E(\theta | \theta \leq p^{\tfrac13})= \frac{\int_{0}^{p^{\frac13}}x dF(x)}{F(p^{\frac13})} = \frac{\left[ \tfrac13 x^2 \right]^{p^{\frac13}}_0}{p^{\frac13}} = \tfrac12 p^{\frac13} \, .$$

$$E(\theta | p) = p \Rightarrow \tfrac12 p^{\frac13} = p \Rightarrow p^*=0 \ \ or \ \ p^*= \frac{\sqrt{2}}{4}$$

There are 2 equilibria.

\subsection*{(e) Are any of the preceding outcomes Pareto efficient? Describe Pareto improvements whenever
possible.}

In case (b), $p=\tfrac12$ is Pareto-efficient. Increasing $p$ in this case, make no difference (in expected value) for the buyers, but is better for sellers that sell.

\bigskip

\section{Jehle \& Reny pp.424, exercise 8.16 - Moral hazard}

Consider the following principal–agent problem. The owner of a firm (the principal) employs a
worker (the agent). The worker can exert low effort, $e = 0$, or high effort, $e = 1$. The resulting
revenue, $r$, to the owner is random, but is more likely to be high when the worker exerts high effort.
Specifically, if the worker exerts low effort, $e = 0$, then


\begin{equation}
r =
    \begin{cases}
0, \text{with probability 2/3}
4, \text{with probability 1/3}
    \end{cases}
\nonumber
\end{equation}
	

If instead the worker exerts high effort, $e = 1$, then

\begin{equation}
r =
    \begin{cases}
0, \text{with probability 1/3}
4, \text{with probability 2/3}
    \end{cases}
\nonumber
\end{equation}


The worker's von Neumann-Morgenstern utility from wage $w$ and effort $e$ is $u(w, e) = \sqrt{w} -e$
The firm's profits are $\pi = r - w$ when revenues are $r$ and the worker's wage is w. A wage contract
$(w_0, w_4)$ specifies the wage, $w_r \ge 0$, that the worker will receive if revenues are $r \in \{0, 4\}$. When
working, the worker chooses effort to maximise expected utility and always has the option (his only
other option) of quitting his job and obtaining $(w, e) = (0, 0)$.

Find the wage contract $(w_0, w_4) \in [0,\infty)^2$ that maximises the firm's expected profits in each
of the situations below



\subsection*{(a)}The owner can observe the worker’s effort and so the contract can also be conditioned
on the effort level of the worker. How much effort does the worker exert in the expected
profit-maximising contract?

\bigskip

Since the firm does not need to provide incentives if effort is observable, the firm need only satisfy the participation constraint (with equality). Since the firm is risk-neutral and the worker is risk-averse, the firm will offer a fixed wage, $w(0)$ in the case of low effort ($e=0$) and $w(1)$ in the case of high effort ($e=1$).

For $e=0$: $\pi_0(0)\sqrt{w(0)} + \pi_4(0)\sqrt{w(0)} - 0 = 0$. Hence, $w(0) = 0$, and the profit for the firm is: $\pi_0(0)0 + \pi_4(0)4 - 0 = \tfrac23 \cdot 0 + \tfrac13 \cdot 4 = \tfrac43$.

For $e=1$: $\pi_0(1)\sqrt{w(1)} + \pi_4(1)\sqrt{w(1)} - 1 = 0$. Hence, $w(1) = 1$ and the profit for the firm is: $\pi_0(1)0 + \pi_4(1)4 - 1 = \tfrac13 \cdot 0 + \tfrac23 \cdot 4 - 1= \tfrac83 - 1 = \tfrac53$.

Since $\tfrac43 < \tfrac53$, the worker will exert high effort ($e=1$) in the expected profit-maximizing contract.

\subsection*{(b)}The owner cannot observe the worker’s effort and so the contract cannot be conditioned on
effort. How much effort does the worker exert in the expected profit-maximising contract now?

\bigskip

Even if the effort is not observable, the firm need not provide incentives in a contract where the worker exert low effort ($e=0$). Hence, $w_0(0) = w_4(0) = 0$, and the profit for the firm is: $\pi_0(0)0 + \pi_4(0)4 - 0 = \tfrac23 \cdot 0 + \tfrac13 \cdot 4 = \tfrac43$.

If the effort is not observable, the firm must provide incentives in a contract where the worker exert high effort ($e=1$). The optimal contract is found by satisfying both the participation constraint and the incentive constraint:

\begin{gather*}
  \pi_0(1)\sqrt{w_0(1)} + \pi_4(1)\sqrt{w_4(1)} - 1 \geq 0  \, ,\\
  \pi_0(1)\sqrt{w_0(1)} + \pi_4(1)\sqrt{w_4(1)} - 1 \geq \pi_0(0)\sqrt{w_0(1)} + \pi_4(0)\sqrt{w_4(1)} \, .
\end{gather*}

Inserting the probabilities we obtain:

\begin{gather*}
  \tfrac13 \sqrt{w_0(1)} + \tfrac23 \sqrt{w_4(1)} \geq 1  \, ,\\
   -\tfrac13 \sqrt{w_0(1)} + \tfrac13 \sqrt{w_4(1)} \geq 1  \, .
\end{gather*}

The lowest wage rates that satisfy these inequalities are $w_0(1) = 0$ (since wage is non-negative) and $w_4(1) = 9$, and the profit for the firm is: $\pi_0(1)0 + \pi_4(1) \left( 4 - 9 \right) = \tfrac13 \cdot 0 - \tfrac23 \cdot 5 = - \tfrac{10}3$.

Since $\tfrac43 > - \tfrac{10}6$, the worker will exert low effort ($e=0$) in the expected profit-maximizing contract when effort is not observable.




\end{document}
