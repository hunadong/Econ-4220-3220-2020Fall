\documentclass{article}
\usepackage[utf8]{inputenc}
\usepackage[T1]{fontenc}
\usepackage[english]{babel}
\setlength{\parindent}{0pt}
\usepackage{hyperref}
\hypersetup{
    colorlinks=true,
    linkcolor=blue,
    filecolor=magenta,      
    urlcolor=cyan}
\usepackage{graphicx}
\graphicspath{ {./pic/} }
\usepackage{multicol}
\usepackage{lscape}

\usepackage{fourier,amssymb,microtype,amsmath,gensymb}
\newcommand{\R}{\mathbb{R}}
\usepackage{mdframed,caption,xcolor}
\usepackage{tikz,tkz-euclide}

\title{Seminar 11. Adverse selection and moral hazard}
\author{Xiaoguang Ling \\  \href{xiaoguang.ling@econ.uio.no}{xiaoguang.ling@econ.uio.no}}
\date{\today}

\begin{document}

\maketitle

%%%%%%%%%%%%%%%%%%%%%%%%%%%%%%%%%%%%%%%%%%%%%%%%%%%%%%%%%%%%%%%%%%%%%%%%%%%%%%%%%%%%%%%%%%%%%%

\section{Jehle \& Reny pp.422, exercise 8.7 - Asymmetric information}

(Akerlof) Consider the following market for used cars. There are many sellers of used cars. Each
seller has exactly one used car to sell and is characterised by the quality of the used car he wishes to
sell. Let $\theta \in [0, 1]$ index the quality of a used car and assume that $\theta$ is \textbf{uniformly distributed on $[0, 1]$}.
If a seller of type $\theta$ sells his car (of quality $\theta$) for a price of $p$, his utility is $u_s(p, \theta)$. If he does not sell
his car, then his utility is 0. Buyers of used cars receive utility $\theta - p$ if they buy a car of quality $\theta$ at
price p and receive utility 0 if they do not purchase a car. There is asymmetric information regarding
the quality of used cars. Sellers know the quality of the car they are selling, but buyers do not know
its quality. Assume that \textbf{there are not enough cars to supply all potential buyers}.


\subsection*{(a) Argue that in a competitive equilibrium under asymmetric information, we must have
$E(\theta | p) = p$}
\begin{mdframed}[backgroundcolor=blue!20,linecolor=white]
See section on ``Asymmetric Information and Adverse Selection'',Jehle \& Reny pp.~382--385.
\end{mdframed}

\begin{itemize}
\item Seller will sell the car only if $u_s(p, \theta) \ge 0$;
\item Buyer will buy the car only if $u_b =\theta - p \ge 0$
\end{itemize}

Therefore in the market, we must have:

\begin{equation}
    \begin{cases}
u_s(p, \theta) \ge 0 \\
u_b = \theta - p \ge 0
    \end{cases}
\nonumber
\end{equation}

However, since the buyer can't observe the quality $\theta$, he/she must guess. The expected quality is:

$$E[\theta|u_s(p, \theta) \ge 0 ]$$

\begin{mdframed}[backgroundcolor=blue!20,linecolor=white]
Be careful, the smart buyer knows only when $u_s(p, \theta) \ge 0$ will the seller sell the car, therefore the expected
quality is $E[\theta|u_s(p, \theta) \ge 0 ]$, not simply $E[\theta]$. 
\end{mdframed}

Therefore, we must have 

\begin{equation}
    \begin{cases}
u_s(p, \theta) \ge 0 \\
E[u_b] = E[\theta|u_s(p, \theta) \ge 0 ] - p \ge 0
    \end{cases}
\nonumber
\end{equation}

Note that we assume \textbf{there are not enough cars to supply all potential buyers},
if $E[\theta|u_s(p, \theta) \ge 0 ] - p \ge 0$, there will be infinite demand, which can't be a equilibrium.
We thus have

$$E[\theta|u_s(p, \theta) \ge 0 ] = p$$

We can write it as $E[\theta|p \ \ s.t. \ \ u_s(p, \theta) \ge 0 ] = p$, or just $E[\theta|p] = p$

%***************************************************
\subsection*{(b) Show that if $u_s(p, \theta) = p - \theta/2$, then every $p \in (0, 1/2]$ is an equilibrium price.}

\begin{mdframed}[backgroundcolor=blue!20,linecolor=white]

Recall: for a random variable $\theta \in [a,b] \subset \R_1$, $F(x) = Pr(\theta\le x)$ is called cummulative distribution function (CDF). $f(x)=F'(x)$ is the probability density function (PDF) of $\theta$. We have:

$$E(\theta) = \int^b_a x dF(x)$$

Similarly, for a number $m \in (a,b)$

$$E(\theta|\theta \le m) = \int^m_a x d F(x|\theta \le m)$$

Where $F(x|\theta \le m) = Pr(\theta \le x | \theta \le m)$. 

\medskip 

Obviously, when $x \in [m,b]$, $Pr(\theta \le x | \theta \le m)=1$; when $x \in [a,m]$, according to Bayes' rule,


 $$Pr(\theta \le x | \theta \le m)= \frac{Pr(\theta \le m | \theta \le x) Pr(\theta \le x)}{Pr(\theta \le m)} = \frac{1 \times F(x)}{F(m)}$$

Therefore,

\begin{equation}
F(x|\theta \le m)=
    \begin{cases}
\frac{F(x)}{F(m)} \quad if \quad x \in [a,m] \\
\ \ 1 \quad \quad if \quad x \in [m,b] 
    \end{cases}
\nonumber
\end{equation}

Thus,

$$E(\theta|\theta \le m) = \int^m_a x d F(x|\theta \le m)=\int^m_a x d \frac{F(x)}{F(m)}$$
\end{mdframed}

Let $p \in (0,\tfrac12]$. Sellers with quality $\theta$ satisfying $u_s(p,\theta)=p - \tfrac{\theta}2 \geq 0$ sell their cars, or equivalently:$$\theta \leq 2p \, .$$

\medskip

Sine $\theta$ is uniformed distributed in $[0,1]$, the CDF of $\theta$ is $F(x) = x$

\medskip

Then:
\begin{align*}
E(\theta | p) = E(\theta | \theta \leq 2p) &= \int_{0}^{2p}x\frac{dF(x)}{F(2p)} \\
&= \int_{0}^{2p} x\frac{dx}{2p} \\
&= \frac{\left[ \tfrac12 x^2 \right]^{2p}_0}{2p} \\
&= \frac{\tfrac12 (2p)^2 - \tfrac12 (0)^2}{2p} \\
&= p
\end{align*}

So $E(\theta | p) = p$ is satisfied for each $p \in (0,\tfrac12]$.

%***************************************************
\subsection*{(c) Find the equilibrium price when $u_s(p, \theta) = p - \sqrt{\theta}$. Describe the equilibrium in words. In
particular, which cars are traded in equilibrium?}

Consider any $p \in [0,1]$. Sellers with quality $\theta$ satisfying $p - \theta^\frac12 \geq 0$ sell their car, or equivalently:$$\theta \leq p^2 \, .$$

Then:
\begin{align*}
E(\theta | p) = E(\theta | \theta \leq p^2) &= \frac{\int_{0}^{p^2}x dF(x)}{F(p^2)} \\
&= \frac{\int_{0}^{p^2}x dx}{p^2} \\
&= \frac{\left[ \tfrac12 \theta^2 \right]^{p^2}_0}{p^2} \\
&= \tfrac12 p^2
\end{align*}

According to the equilibrium result we found in part (a), i.e.
$E(\theta | p) = p$, in an equilibrium we must have:

$$\tfrac12 p^2 = p \Rightarrow (p-2)p = 0 \Rightarrow p^* = 0$$ 

We also have $\theta \leq p^2 \Rightarrow \theta^* = 0$

\medskip

Only $\theta = 0$ will be traded at $p=0$. The market breaks down completely.
%***************************************************

\subsection*{(d) Find an equilibrium price when $u_s(p, \theta) = p - \theta^3$. How many equilibria are there in this case?  }

Consider any $p \in [0,1]$. Sellers with quality $\theta$ satisfying $p - \theta^3 \geq 0$ sell their car, or equivalently:$$\theta \leq p^{\tfrac13} \, .$$
Then:
$$E(\theta | p) = E(\theta | \theta \leq p^{\tfrac13})= \frac{\int_{0}^{p^{\frac13}}x dF(x)}{F(p^{\frac13})} = \frac{\left[ \tfrac13 x^2 \right]^{p^{\frac13}}_0}{p^{\frac13}} = \tfrac12 p^{\frac13} \, .$$

$$E(\theta | p) = p \Rightarrow \tfrac12 p^{\frac13} = p \Rightarrow p^*=0 \ \ or \ \ p^*= \frac{\sqrt{2}}{4}$$

There are 2 equilibria.
%***************************************************

\subsection*{(e) Are any of the preceding outcomes Pareto efficient? Describe Pareto improvements whenever
possible.}

\textbf{In case (b)}, $p=\tfrac12$ is Pareto-efficient. For any $p \in (0, \tfrac12]$, we have $$E[u_b] = E[\theta|u_s(p, \theta) \ge 0 ] - p= 0$$
while for the seller, since $u_s(p, \theta) = p - \theta/2$, higher $p^*$ means higher payoff (and all sellers can sell the car now).

\textbf{In case (c)}, the equilibrium condition requires $p^*=0$, there can't be any improvement.

\textbf{In case (d)}, there are 2 equilibria. Similar to case (b), $p^*=\frac{\sqrt{2}}{4}$ is a Pareto improvement comparing
with $p^*=0$ (again, only the seller benefits).


\bigskip
%%%%%%%%%%%%%%%%%%%%%%%%%%%%%%%%%%%%%%%%%%%%%%%%%%%%%%%%%%%%%%%%%%%%%%%%%%%%%%%%%%%%%%%%%%%%%%

\section{Jehle \& Reny pp.424, exercise 8.16 - Moral hazard}

Consider the following principal-agent problem. The owner of a firm (the principal) employs a
worker (the agent). The worker can exert low effort, $e = 0$, or high effort, $e = 1$. The resulting
revenue, $r$, to the owner is random, but is more likely to be high when the worker exerts high effort.
Specifically, if the worker exerts low effort, $e = 0$, then


\begin{equation}
r =
    \begin{cases}
0, \text{with probability 2/3} \\
4, \text{with probability 1/3}
    \end{cases}
\nonumber
\end{equation}
	

If instead the worker exerts high effort, $e = 1$, then

\begin{equation}
r =
    \begin{cases}
0, \text{with probability 1/3} \\
4, \text{with probability 2/3}
    \end{cases}
\nonumber
\end{equation}


The worker's von Neumann-Morgenstern utility from wage $w$ and effort $e$ is $u(w, e) = \sqrt{w} -e$.
The firm's profits are $\pi = r - w$ when revenues are $r$ and the worker's wage is $w$. A wage contract
$(w_0, w_4)$ specifies the wage, $w_r \ge 0$, that the worker will receive if revenues are $r \in \{0, 4\}$. When
working, the worker chooses effort to maximise expected utility and always has the option (his only
other option) of quitting his job and obtaining $(w, e) = (0, 0)$.

\medskip

Find the wage contract $(w_0, w_4) \in [0,\infty)^2$ that maximises the firm's expected profits in each
of the situations below.


%***************************************************
\begin{mdframed}[backgroundcolor=blue!20,linecolor=white]
(Read also Watson pp.340-345)

In a principal-agent problem, the principal usually wants to 
solve 2 problems by designing a contract:

\begin{itemize}
\item Participation Constraint (PC): the agent will agree to work for the principal only if $u_a(agree) \ge u_a(reject)$.
\item Incentive Compatibility (IC):the agent will pay effort only if $u_a(effort=1) \ge u_a(effort=0)$.
\end{itemize}

When the agent's effort is observable, the principal can set wage depending on the effort level (pay directly for the effort). Therefore Incentive Compatibility is not a problem.

\medskip

When the agent's effort is unobservable, the agent can pretend 
to pay effort (Moral Hazard); while the principal can either allow the agent to pay no effort (constant low wage) or make a wage plan depending on the outcome to achieve Incentive Compatibility.


\end{mdframed}

\bigskip

\subsection*{(a) Worker's effort type can be observed}
The owner can observe the worker's effort and so the contract can also be conditioned
on the effort level of the worker. How much effort does the worker exert in the expected
profit-maximising contract?

\bigskip

Denote $w_r(e)$ as the wage plan given outcome $r \in \{0,4\}$
and effort level $e\in \{0,1\}$. For example, $w_4(0)$ is the wage in the case of low effort ($e=0$) but the outcome is high ($r=4$).

\medskip

There are 4 possible wages depending on the efforts and outcome:

\begin{itemize}
\item If low effort: $w_0(0),w_4(0)$
\item If high effort: $w_0(1),w_4(1)$
\end{itemize}

Let $p = Pr(r=0), 1-p = Pr(r=4)$. Now consider the following 2 wage plans:

\begin{itemize}
\item Uncertain wage: $w_0$ if outcome is $0$; $w_4$ if outcome is $4$.

Denote $w^f = pw_0 +(1-p)w_4$
\item Fixed wage: no matter what the outcome is, pay $w^f$
\end{itemize}


We can see the firm is risk neutral while the worker is risk averse w.r.t. the wage. That is to say, the firm is indifferent to the uncertain wage and the fixed wage, while the worker prefers fixed wage $w^f$ than uncertain wage $pw_0 +(1-p)w_4$.

\medskip

To minimize the cost, the firm will choose a fixed wage, otherwise the firm must compensate the worker for the risk(recall risk premium and certainty equivalent), i.e. $w_0(0)=w_4(0)=w(0), w_0(1)=w_4(1)=w(1)$

\begin{mdframed}[backgroundcolor=blue!20,linecolor=white]
recall seminar 4:
\begin{itemize}
\item  $\pi = r - w \Rightarrow -\frac{\pi''}{w'} = 0$
\item $u(w, e) = \sqrt{w} -e \Rightarrow -\frac{u''}{w'} = -\frac{-0.25w^{-1.5}}{0.5w^{-0.5}} = 0.5w^{-1}>0$
\end{itemize}
\end{mdframed}

Participation Constraint (PC) requires:

\begin{itemize}
\item If effort$=0$, $u(agree) \ge u(reject) \Rightarrow \sqrt{w(0)} \ge 0 \Rightarrow w(0) \ge 0$;
\item If effort$=1$, $u(agree) \ge u(reject) \Rightarrow \sqrt{w(1)} -1 \ge 0 \Rightarrow w(1) \ge 1$;
\end{itemize}

To minimize the cost, the firm will let $w(0) = 0,  w(1) = 1$.

\medskip

The expected profit of the firm is 
\begin{itemize}
\item If effort$=0$, $E[\pi_0]= \tfrac23 \times 0 + \tfrac13 \times 4 - 0 =\tfrac43$
\item If effort$=1$, $E[\pi_1]= \tfrac13 \times 0 + \tfrac23 \times 4 - 1 =\tfrac53$
\end{itemize}

The firm prefers the worker to pay effort. In the expected
profit-maximising contract the worker will exert effort $=1$.

\begin{mdframed}[backgroundcolor=blue!20,linecolor=white]
The firm can promise offering a $wage = 1$ if the worker exerts efforts. Since the effort can be observed, the promise is realistic. And there is no incentive for the worker to reject.
\end{mdframed}

%***************************************************
\subsection*{(b) Worker's effort type can't be observed}
The owner cannot observe the worker's effort and so the contract cannot be conditioned on
effort. How much effort does the worker exert in the expected profit-maximising contract now?

\bigskip

\textbf{(1) Case 1: the firm allows the worker exerting low effort.}

\medskip

If the firm still provides a fixed wage, since no matter what the outcome is the worker is paid the same, the worker will not pay any effort (Moral Hazard). If the firm prefers the worker exerting no effort, then the only problem is Participation Constraint, the result is the same as part (a) when $e=0$:

$$w_0 = w_4 = 0$$

The expected profit for the firm is: $$E[\pi]= \tfrac23 \times 0 + \tfrac13 \times 4 - 0 =\tfrac43$$

\medskip

\textbf{(2) Case 2: the firm wants the worker exerting high effort.}

\medskip

If the effort is not observable, the firm must provide incentives in a contract where the worker exerts high effort ($e=1$). The optimal contract is found by satisfying both the Participation Constraint and the Incentive Compatibility:

\begin{gather*}
 \text{PC:} \quad \tfrac13\sqrt{w_0} + \tfrac23\sqrt{w_4} - 1 \geq 0  \, ,\\
 \text{IC:} \quad   \tfrac13\sqrt{w_0} + \tfrac23\sqrt{w_4} - 1 \geq \tfrac23\sqrt{w_0} + \tfrac13\sqrt{w_4} -0 \, .
\end{gather*}

Simplifying:

\begin{gather*}
 \text{PC:}   \tfrac13 \sqrt{w_0} + \tfrac23 \sqrt{w_4} \geq 1  \, ,\\
 \text{IC:}    -\tfrac13 \sqrt{w_0} + \tfrac13 \sqrt{w_4} \geq 1  \, .
\end{gather*}

\begin{mdframed}[backgroundcolor=blue!20,linecolor=white]
The IC can be written as $\sqrt{w_4} \geq  \sqrt{w_0} + 3$. The higher $w_0$ is, the higher $w_4$ must be. Therefore we let $w_0=0$ and the lowest $w_4=9$. They also satisfy the PC condition.
\end{mdframed}

The lowest (non-negative) wage rates that satisfy these inequalities are $w_0 = 0$ and $w_4 = 9$.


The expected profit for the firm is: 
$$ E[\pi]=\tfrac13 \times (0-0) + \tfrac23 \times (4 -9) = - \tfrac{10}3$$

Since $\tfrac43 > - \tfrac{10}3$, the firm prefers the worker to exert low effort. Thus the worker will exert low effort ($e=0$) in the expected profit-maximizing contract when effort is not observable.

\begin{mdframed}[backgroundcolor=blue!20,linecolor=white]
In this case, the firm simply provides a fixed wage 0 and the worker will not exert high effort. There is no incentive to deviate.
\end{mdframed}



\end{document}
