\documentclass{article}
\usepackage[utf8]{inputenc}
\usepackage[T1]{fontenc}
\usepackage[english]{babel}
\setlength{\parindent}{0pt}
\usepackage{hyperref}
\hypersetup{
    colorlinks=true,
    linkcolor=blue,
    filecolor=magenta,      
    urlcolor=cyan}
\usepackage{graphicx}
\graphicspath{ {./pic/} }
\usepackage{multicol}
\usepackage{lscape}

\usepackage{fourier,amssymb,microtype,amsmath,gensymb}
\newcommand{\R}{\mathbb{R}}
\usepackage{mdframed,caption,xcolor}
\usepackage{tikz,tkz-euclide}

\title{Seminar 6. Walrasian Equilibrium in a Barter Economy}
\author{Xiaoguang Ling \\  \href{xiaoguang.ling@econ.uio.no}{xiaoguang.ling@econ.uio.no}}
\date{\today}

\begin{document}

\maketitle

%%%%%%%%%%%%%%%%%%%%%%%%%%%%%%%%%%%%%%%%%%%%%%%%%%%%%%%%%%%%%%%%%%%%%%%%%%%%%%%%%%%%%%%%%%%%%%


%%%%%%%%%%%%%%%%%%%%%%%%%%%%%%%%%%%%%%%%%%%%%%%%%%%%%%%%%%%%%%%%%%%%%%%%%%%%%%%%%%%%%%%%%%%%%%
\section{Jehle \& Reny 5.4 - Excess demand function and GE}

Derive the excess demand function $z(p)$ for the economy in Example 5.1. Verify that it satisfies
Walras' law.

\begin{mdframed}[backgroundcolor=blue!20,linecolor=white]

Suppose we have a good-exchange economy,

\begin{itemize}
\item $I$ is the set of all the individuals (consumers) in the economy,
\item The prices of all $n$ commodities is expressed by a vector $p = (p_1, p_2, \dots, p_n)$,
\item Every consumer has some endowments in the form of commodities expressed by a vector $e^i = (e^i_1,e^i_2,\dots,e^i_n)$,
\item $p\cdot e^i$ is the income of consumer $i$,
\end{itemize}

Assume (Assumption 5.1 on pp.203) that every consumer has a utility function $u^i$, which is continuous, strongly increasing, and strictly quasiconcave on $\R^n_{+}$.

\begin{itemize}
\item By solving consumer $i$'s' utility maximization problem, consumer $i$'s Marshallian demand function is $x^i(p,p\cdot e^i) = (x^i_1,x^i_2,\dots,x^i_n)$
\end{itemize}

\textbf{General Equilibrium}: When demand equal to supply in \textbf{every market} (market for every commodity), we would say that the system of markets is in General Equilibrium.

\vspace{2mm}

We use \textbf{Excess Demand} to describe "demand equal to supply".
\vspace{2mm}

\textbf{DEFINITION 5.4 Aggregate Excess Demand} (Jehle \& Reny pp.204)

The aggregate excess demand function for good $k$ is the real-valued function,
$$z_k(p) \equiv \Sigma_{i \in I } x^i_k(p,p\cdot e^i) - \Sigma_{i \in I } e^i_k$$

Where,
\begin{itemize}
\item $\Sigma_{i \in I } x^i_k(p,p\cdot e^i)$ is the summation of all consumers' Marshallian demand for commodity $k$,
\item $\Sigma_{i \in I } e^i_k$ is the total amount of commodity $k$ in this economy.
\end{itemize}

When $z_k(p) > 0$, the aggregate demand for good $k$ exceeds the aggregate endowment of good $k$ and so there is excess demand for good k. When $z_k(p) < 0$, there is excess supply of good $k$. That's why $z_k(p)$ is called "Excess Demand" for $k$.
\vspace{2mm}

The \textbf{aggregate excess demand function} is a vector-valued function,
$$z_(p) \equiv [z_1(p),z_2(p),\dots,z_n(p)]$$

\vspace{2mm}


When $\exists$ $p^* \in \R^n_{++}$ s.t. $z(p^*) = 0$, we say
Walrasian Equilibrium (WE) exists. A WE in a barter economy includes a price vector $p^*$ and an allocation (e.g. Marshallian demand) vector $x(p^*,p^*\cdot e)$.

\vspace{4mm}

THEOREM 5.2 Properties of Aggregate Excess Demand Functions (pp.204)

If for each consumer $i$, $u^i$ satisfies Assumption 5.1, then for all $p \gg 0$,


\begin{enumerate}
\item Continuity: $z(.)$ is continuous at $p$.
\item Homogeneity: $z(\lambda p) = z(p) \ \ \forall \lambda > 0$.
\item Walras' law: $p \cdot z(p) = 0$.
\end{enumerate}

\vspace{4mm}

\textbf{THEOREM 5.5 Existence of Walrasian Equilibrium}

If each consumer's utility function satisfies Assumption 5.1, and
$\Sigma_{i=1}^{I} e^i \gg 0$, then there exists at least one price vector, $p^* \gg 0$, such that $z(p^*)$.


\end{mdframed}


\vspace{2mm}

\begin{mdframed}[backgroundcolor=blue!20,linecolor=white]
\textbf{Example 5.1 on pp.211}

In a simple two-person economy, consumers 1 and 2 have identical CES utility functions,
$$u^i(x_1,x_2) = x^\rho_1 + x^\rho_2, \ \ \ i= 1,2$$
where $\rho \in (0,1)$.

The initial endowments are $e^1 = (1,0), e^2 = (0,1)$.

\vspace{2mm}

\textbf{Does WE exist?}

Yes. The requirements of Theorem 5.5 are satisfied.

\begin{itemize}
\item $\Sigma_{i=1}^{2} e^i = (1,0) + (0,1) = (1,1) \gg 0$
\item $u^i(x_1,x_2) = x^\rho_1 + x^\rho_2$ is strongly increasing and strictly quasiconcave on $\R^n_{+}$ when $\rho \in (0,1)$
\end{itemize}


\textbf{How to find WE? }

We let $z(p)=0$ to find $p$.

\vspace{2mm}

\textbf{How to find WEA? }

By substituting $p^*$ and $y^* = p^*e$ into $x(p,y)$.

\end{mdframed}

%***************************************************
\subsection{Excess demand function $z(p)$}


From Example 1.11 on pp.26, we know the Marshallian demands of consumer $i$ for commodity $1$ and commodity $2$ are:

$$x^i_1(p,y^i) = \frac{p^{r-1}_1y^i}{p_1^r+p_2^r},$$
$$x^i_2(p,y^i) = \frac{p^{r-1}_2y^i}{p_1^r+p_2^r}.$$

where $r= \frac{\rho}{\rho-1},\ i = 1,2$.

Given any price vector $p = (p_1,p_2)$, and initial endowment $e^1 = (1,0), e^2 = (0,1)$, we know the income of the two consumers are 
$$y^1= p(1,0)' = p_1$$
$$y^2= p(0,1)' = p_2$$

According to Deffinition 5.4, we have aggregated excess demand for commodity $1$:

\begin{align*}
z_1(p) &= \Sigma_{i=1}^2 x^i_1(p,p\cdot e^i) - \Sigma_{i=1}^2 e^i_1 \\
&= [x^1_1(p,p_1) + x^2_1(p,p_2)] -  (e^1_1+e^2_1) \\
&= ( \frac{p^{r-1}_1 p_1}{p_1^r+p_2^r} + \frac{p^{r-1}_1 p_2}{p_1^r+p_2^r}) - (1+0) \\
&= \frac{p^{r-1}_1 (p_1+p_2)}{p_1^r+p_2^r} -1
\end{align*}

Similarly, the aggregated excess demand for commodity $2$ is:

\begin{align*}
z_2(p) &= \Sigma_{i=1}^2 x^i_2(p,p\cdot e^i) - \Sigma_{i=1}^2 e^i_2 \\
&= [x^1_2(p,p_1) + x^2_2(p,p_2)] -  (e^1_2+e^2_2) \\
&= ( \frac{p^{r-1}_2 p_1}{p_1^r+p_2^r} + \frac{p^{r-1}_2 p_2}{p_1^r+p_2^r}) - (1+0) \\
&= \frac{p^{r-1}_2 (p_1+p_2)}{p_1^r+p_2^r} -1
\end{align*}

Thus, the  \textbf{Aggregated Excess Demand Function} is vector:

$$z(p) = (z_1(p),z_2(p)) = (\frac{p^{r-1}_1 (p_1+p_2)}{p_1^r+p_2^r} -1,\frac{p^{r-1}_2 (p_1+p_2)}{p_1^r+p_2^r} -1)$$


\begin{mdframed}[backgroundcolor=yellow!20,linecolor=white]
Note Aggregated Excess Demand Function $z(p)$ is a vector, and each element corresponds with one commodity.
\end{mdframed}

%***************************************************
\subsection{Walras' law}

\begin{mdframed}[backgroundcolor=blue!20,linecolor=white]
\begin{itemize}
\item Walras' law: $p \cdot z(p) = 0$.
\end{itemize}
\end{mdframed}

\begin{align*}
p \cdot z(p) &= (p_1,p_2)(z_1(p),z_2(p))' \\
&= p_1 [\frac{p^{r-1}_1 (p_1+p_2)}{p_1^r+p_2^r} -1] + p_2 [\frac{p^{r-1}_2 (p_1+p_2)}{p_1^r+p_2^r} -1] \\
&= [\frac{p^r_1 (p_1+p_2)}{p_1^r+p_2^r} -p_1] + [\frac{p^{r}_2 (p_1+p_2)}{p_1^r+p_2^r} -p_2]  \\
&= \frac{(p^r_1+p^r_2) (p_1+p_2)}{p_1^r+p_2^r} -p_1 - p_2 \\
&=0
\end{align*}




%%%%%%%%%%%%%%%%%%%%%%%%%%%%%%%%%%%%%%%%%%%%%%%%%%%%%%%%%%%%%%%%%%%%%%%%%%%%%%%%%%%%%%%%%%%%%%
\section{Jehle \& Reny 5.5 - WEA and Edgeworth box}

In Example 5.1, calculate the consumers' Walrasian equilibrium allocations and illustrate in an
Edgeworth box. Sketch in the contract curve and identify the core.


%***************************************************
\subsection{WEA}

We already have $z(p)$. Now let $z(p)=0$ to find $p^*$.

When $ (z_1(p),z_2(p)) = (0,0)$, we have:
(Note I omitted star below for simplicity, but you should know only $p^*$ leads to $z(p)=0$)
$$\frac{p^{r-1}_1 (p_1+p_2)}{p_1^r+p_2^r} =1, \ \ \frac{p^{r-1}_2 (p_1+p_2)}{p_1^r+p_2^r} =1$$

For the first commodity:

\begin{align*}
\frac{p^{-r}_1}{p^{-r}_1}\frac{p^{r-1}_1 (p_1+p_2)}{p_1^r+p_2^r} &=1 \\
\frac{p^{-1}_1 (p_1+p_2)}{(p_1/p_1)^r+(p_2/p_1)^r} &=1 \\
\frac{1+p_2/p_1}{1+(p_2/p_1)^r} &=1 \\
1+\frac{p_2}{p_1} &=1+(\frac{p_2}{p_1})^r \\
(\frac{p_2}{p_1})^{r-1} = 1
\end{align*}

We know $r-1 = \frac{\rho}{\rho-1} -1 = \frac{1}{\rho-1} \ne 0$, $p \gg 0$. Then $\frac{p_2}{p_1} = 1$

Similarly, for the second commodity, we have  $\frac{p_1}{p_2} = 1$

To conclude, the WE price $p^*$ is $(p^*_1.p^*_2) \ s.t.\ p^*_1=p^*_2$. Let's just denote the $p^*_1=p^*_2 = a$, the demands under the price $p^*$ are:

$$x^i_1(p,a) = \frac{a^{r-1} a}{a^r+a^r} = 0.5 ,$$
$$x^i_2(p,a) = \frac{a^{r-1} a}{a^r+a^r} = 0.5.$$
$i = 1,2$. The WEA is thus $x^* = ((0.5,0.5),(0.5,0.5))$
\begin{mdframed}[backgroundcolor=blue!20,linecolor=white]

\begin{itemize}
\item Only relative price $\frac{p_1}{p_2}$ matters, since you can always "rescale" the prices;
\item To describe WE, you need to denote both $p^*$ and WEA.
\end{itemize}
\end{mdframed}



%***************************************************
\subsection{Edgeworth box}

\begin{mdframed}[backgroundcolor=blue!20,linecolor=white]
\textbf{Contract curve} The curve that links the two consumers' indifference curves' tangent point.

\textbf{Core} Given some endowment $e$, the core of the economy is the set of all feasible allocations that are not against ("blocked") by any consumers (a formal definition is on pp.200-201).

\begin{center}
\begin{tikzpicture}[scale=0.7]
\draw [->,blue] (0,0) node [below] {0}  -- (8,0) node [below] {$x^1_1$};
\draw [->,blue] (0,0)  -- (0,8) node [left] {$x^1_2$};

\draw [blue] (1,5) to [out=-70, in=145] (3.5,1.5) to [out=-35, in=165] (7,0);
\draw [blue] (3.5,5.6) to [out=-70, in=145] (4.68,3.64) to [out=-35, in=165] (6.5,2.7);

\draw  (0,7) -- (7,0);
\node[below] at (3.5,3) {\tiny{Budget}};

\node[left] at (0,7) {\tiny{$\frac{y}{p_1} = \frac{p_1}{p_2}$}};
\node[below] at (6.5,0) {\tiny{$\frac{y}{p_1} = \frac{p_1}{p_1} =1$}};


\node[above] at (7.8,0) {$e^1=(1,0)$};
\draw[fill,yellow] (7,0) circle [radius =0.05];

\end{tikzpicture}
\captionof{figure}{Consumer 1}
\label{fig:c1}
\end{center}
\end{mdframed}

\begin{center}
\begin{tikzpicture}[scale=1.1]
\draw [->,blue] (0,0) node [below] {0}  -- (8,0) node [below] {$x^1_1$};
\draw [->,blue] (0,0)  -- (0,8) node [left] {$x^1_2$};

\draw [->,red] (7,7) node [right] {0} -- (-1,7) node [below] {$x^2_1$};
\draw [->,red] (7,7) -- (7,-1) node [left] {$x^2_2$};


\draw [blue] (1,5) to [out=-70, in=145] (3.5,1.5) to [out=-35, in=165] (7,0);
\draw [blue] (3.5,5.6) to [out=-70, in=145] (4.68,3.64) to [out=-35, in=165] (6.5,2.7);

\draw [red] (2.8,4.7) to [out=-20, in=135] (4.9,3.4) to [out=-45, in=115] (7,0);

\draw plot [smooth] coordinates {(1,1) (3.8,2.8) (6,6)};
\draw [ultra thick] plot [smooth] coordinates {(2.8,2.05) (3.8,2.8) (4.55,3.75)};

\node [right] at (6,6) {\scriptsize $C'$};
\node [left] at (1,1) {\scriptsize $C$};
\draw[fill] (4.55,3.75) circle [radius =0.03];

\node[above] at (7.6,0) {$e=(1,1)$};
\draw[fill,yellow] (7,0) circle [radius =0.05];

\draw  (0,7) -- (7,0);

\end{tikzpicture}
\captionof{figure}{The core}
\label{fig:c1}
\end{center}


%%%%%%%%%%%%%%%%%%%%%%%%%%%%%%%%%%%%%%%%%%%%%%%%%%%%%%%%%%%%%%%%%%%%%%%%%%%%%%%%%%%%%%%%%%%%%%
\section{Jehle \& Reny 5.11 - Pareto-efficient allocations and WEA}

Consider a two-consumer, two-good exchange economy. Utility functions and endowments are

$$u^1(x_1,x_2) = (x_1x_2)^2 \ \ \ and \ \ \  e^1 = (18,4)$$
$$u^2(x_1,x_2) = ln(x_1) + 2 ln(x_2) \ \ \ and \ \ \  e^2 = (3,6)$$

\begin{enumerate}
\item Characterise the set of Pareto-efficient allocations as completely as possible.
\item Characterise the core of this economy.
\item Find a Walrasian equilibrium and compute the WEA.
\item Verify that the WEA you found in part (c) is in the core.
\end{enumerate}

%***************************************************
\subsection{Pareto-efficient allocations}

\begin{mdframed}[backgroundcolor=blue!20,linecolor=white]
\textbf{Pareto-efficient allocations}



\end{mdframed}



%***************************************************
\subsection{Core}




%***************************************************
\subsection{WEA}




%***************************************************
\subsection{WEA is in the core}















\end{document}
