\documentclass{article}
\usepackage[utf8]{inputenc}
\usepackage[T1]{fontenc}
\usepackage[english]{babel}
\setlength{\parindent}{0pt}
\usepackage{hyperref}
\hypersetup{
    colorlinks=true,
    linkcolor=blue,
    filecolor=magenta,      
    urlcolor=cyan}
\usepackage{graphicx}
\graphicspath{ {./pic/} }
\usepackage{multicol}
\usepackage{lscape}

\usepackage{fourier,amssymb,microtype,amsmath,gensymb}
\newcommand{\R}{\mathbb{R}}
\usepackage{mdframed,caption,xcolor}
\usepackage{tikz,tkz-euclide}

\title{Seminar 6. Walrasian Equilibrium in a Barter Economy}
\author{Xiaoguang Ling \\  \href{xiaoguang.ling@econ.uio.no}{xiaoguang.ling@econ.uio.no}}
\date{\today}

\begin{document}

\maketitle

%%%%%%%%%%%%%%%%%%%%%%%%%%%%%%%%%%%%%%%%%%%%%%%%%%%%%%%%%%%%%%%%%%%%%%%%%%%%%%%%%%%%%%%%%%%%%%


%%%%%%%%%%%%%%%%%%%%%%%%%%%%%%%%%%%%%%%%%%%%%%%%%%%%%%%%%%%%%%%%%%%%%%%%%%%%%%%%%%%%%%%%%%%%%%
\section{Jehle \& Reny 5.4 - Excess demand function and GE}

Derive the excess demand function $z(p)$ for the economy in Example 5.1. Verify that it satisfies
Walras' law.

\begin{mdframed}[backgroundcolor=blue!20,linecolor=white]

Suppose we have a good-exchange economy,

\begin{itemize}
\item $I$ is the set of all the individuals (consumers) in the economy,
\item The prices of all $n$ commodities is expressed by a vector $p = (p_1, p_2, \dots, p_n)$,
\item Every consumer has some endowments in the form of commodities expressed by a vector $e^i = (e^i_1,e^i_2,\dots,e^i_n)$,
\item $p\cdot e^i$ is the income of consumer $i$,
\end{itemize}

Assume (Assumption 5.1 on pp.203) that every consumer has a utility function $u^i$, which is continuous, strongly increasing, and strictly quasiconcave on $\R^n_{+}$.

\begin{itemize}
\item By solving consumer $i$'s' utility maximization problem, consumer $i$'s Marshallian demand function is $x^i(p,p\cdot e^i) = (x^i_1,x^i_2,\dots,x^i_n)$
\end{itemize}

\textbf{General Equilibrium}: When demand equal to supply in \textbf{every market} (market for every commodity), we would say that the system of markets is in General Equilibrium.

\vspace{2mm}

We use \textbf{Excess Demand} to describe "demand equal to supply".
\vspace{2mm}

\textbf{DEFINITION 5.4 Aggregate Excess Demand} (Jehle \& Reny pp.204)

The aggregate excess demand function for good $k$ is the real-valued function,
$$z_k(p) \equiv \Sigma_{i \in I } x^i_k(p,p\cdot e^i) - \Sigma_{i \in I } e^i_k$$

Where,
\begin{itemize}
\item $\Sigma_{i \in I } x^i_k(p,p\cdot e^i)$ is the summation of all consumers' Marshallian demand for commodity $k$,
\item $\Sigma_{i \in I } e^i_k$ is the total amount of commodity $k$ in this economy.
\end{itemize}

When $z_k(p) > 0$, the aggregate demand for good $k$ exceeds the aggregate endowment of good $k$ and so there is excess demand for good k. When $z_k(p) < 0$, there is excess supply of good $k$. That's why $z_k(p)$ is called "Excess Demand" for $k$.
\vspace{2mm}

The \textbf{aggregate excess demand function} is a vector-valued function,
$$z_(p) \equiv [z_1(p),z_2(p),\dots,z_n(p)]$$

\vspace{2mm}


When $\exists$ $p^* \in \R^n_{++}$ s.t. $z(p^*) = 0$, we say
Walrasian Equilibrium (WE) exists. A WE in a barter economy includes a price vector $p^*$ and an allocation (e.g. Marshallian demand) vector $x(p^*,p^*\cdot e)$.

\vspace{4mm}

THEOREM 5.2 Properties of Aggregate Excess Demand Functions (pp.204)

If for each consumer $i$, $u^i$ satisfies Assumption 5.1, then for all $p \gg 0$,


\begin{enumerate}
\item Continuity: $z(.)$ is continuous at $p$.
\item Homogeneity: $z(\lambda p) = z(p) \ \ \forall \lambda > 0$.
\item Walras' law: $p \cdot z(p) = 0$.
\end{enumerate}
\end{mdframed}



%%%%%%%%%%%%%%%%%%%%%%%%%%%%%%%%%%%%%%%%%%%%%%%%%%%%%%%%%%%%%%%%%%%%%%%%%%%%%%%%%%%%%%%%%%%%%%
\section{Jehle \& Reny 5.5 - WEA and Edgeworth box}

In Example 5.1, calculate the consumers' Walrasian equilibrium allocations and illustrate in an
Edgeworth box. Sketch in the contract curve and identify the core.


%***************************************************
\subsection{WEA}




%***************************************************
\subsection{Edgeworth box}

\begin{mdframed}[backgroundcolor=blue!20,linecolor=white]
\textbf{Contract curve} The curve that links the two consumers' indifference curves' tangent point.

\textbf{Core} Given some endowment $e$, the core of the economy is the set of all feasible allocations that are not against ("blocked") by any consumers (a formal definition is on pp.200-201).

\end{mdframed}





%%%%%%%%%%%%%%%%%%%%%%%%%%%%%%%%%%%%%%%%%%%%%%%%%%%%%%%%%%%%%%%%%%%%%%%%%%%%%%%%%%%%%%%%%%%%%%
\section{Jehle \& Reny 5.11 - Pareto-efficient allocations and WEA}

Consider a two-consumer, two-good exchange economy. Utility functions and endowments are

$$u^1(x_1,x_2) = (x_1x_2)^2 \ \ \ and \ \ \  e^1 = (18,4)$$
$$u^2(x_1,x_2) = ln(x_1) + 2 ln(x_2) \ \ \ and \ \ \  e^2 = (3,6)$$

\begin{enumerate}
\item Characterise the set of Pareto-efficient allocations as completely as possible.
\item Characterise the core of this economy.
\item Find a Walrasian equilibrium and compute the WEA.
\item Verify that the WEA you found in part (c) is in the core.
\end{enumerate}

%***************************************************
\subsection{Pareto-efficient allocations}

\begin{mdframed}[backgroundcolor=blue!20,linecolor=white]
\textbf{Pareto-efficient allocations}



\end{mdframed}



%***************************************************
\subsection{Core}




%***************************************************
\subsection{WEA}




%***************************************************
\subsection{WEA is in the core}















\end{document}
