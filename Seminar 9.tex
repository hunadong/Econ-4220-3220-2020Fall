\documentclass{article}
\usepackage[utf8]{inputenc}
\usepackage[T1]{fontenc}
\usepackage[english]{babel}
\setlength{\parindent}{0pt}
\usepackage{hyperref}
\hypersetup{
    colorlinks=true,
    linkcolor=blue,
    filecolor=magenta,      
    urlcolor=cyan}
\usepackage{graphicx}
\graphicspath{ {./pic/} }
\usepackage{multicol}
\usepackage{lscape}

\usepackage{fourier,amssymb,microtype,amsmath,gensymb}
\newcommand{\R}{\mathbb{R}}
\usepackage{mdframed,caption,xcolor}
\usepackage{tikz,tkz-euclide}

\title{Seminar 9.Incomplete information in Static Games}
\author{Xiaoguang Ling \\  \href{xiaoguang.ling@econ.uio.no}{xiaoguang.ling@econ.uio.no}}
\date{\today}

\begin{document}

\maketitle

%%%%%%%%%%%%%%%%%%%%%%%%%%%%%%%%%%%%%%%%%%%%%%%%%%%%%%%%%%%%%%%%%%%%%%%%%%%%%%%%%%%%%%%%%%%%%%
\section{Problem 1 - Bayesian normal form}

Consider the following two normal form games. Assume that \textbf{only player $1$ knows} which game is being played, while player $2$ thinks that the two games are \textbf{equally likely}.\vspace{-21pt}

\begin{center}
Game $1$ \vspace{6pt}

$
\begin{array}{c|c|c|}
 & L & R \\
\hline
U & 0,0 & 4,2 \\
\hline
D & 2,6 & 0,8 \\
\hline
\end{array}
$
\end{center}

\begin{center}
Game $2$ \vspace{6pt}

$
\begin{array}{c|c|c|}
 & L & R \\
\hline
U' & 0,2 & 0,0 \\
\hline
D' & 2,0 & 2,2 \\
\hline
\end{array}
$
\end{center}

\subsection*{(a) Model this situation in an ex ante perspective by specifying the Bayesian normal form.}

\begin{mdframed}[backgroundcolor=blue!20,linecolor=white]

Who has contingent strategies?

\begin{itemize}
\item Player 1 has private information (Game 1 or Game 2 is played), therefore has $2 \times 2 = 4$ types of contingent strategies,
\item Player 2 doesn't know either which game is played (incomplete) or how player 1 moves (static), thus can only choose $L$ or $R$.
\end{itemize}
\end{mdframed}

\begin{center}
$
\begin{array}{c|c|c|}
 & L & R \\
\hline
UU' & 0,1 & 2,1 \\
\hline
UD' & 1,0 & 3,2 \\
\hline
DU' & 1,4 & 0,4 \\
\hline
DD' & 2,3 & 1,5 \\
\hline
\end{array}
$
\end{center}
%

\begin{mdframed}[backgroundcolor=blue!20,linecolor=white]

We calculate the payoff according to player 2's belief. For example, the 
payoff for $(UU',L)$ is (0,1):
$$U^1= 0.5 \times 0 +  0.5 \times 0 =0$$
$$U^2= 0.5 \times 0 +  0.5 \times 2 =1$$


Note: the question is slightly different from those in Watson textbook.
\medskip

In the textbook, it often says
the nature moves with probability $(Pr_1,Pr_2)$, i.e. both of the two players have expected payoff.
Here only player 2 has to guess, while player 1 doesn't need to. 
\medskip

Although the question can be solved in the same way, you may wonder if it's reasonable to calculate player 1's "expected payoff" according to player 2's belief.

\medskip

\textbf{The payoff in the Bayesian noraml form is actually player 2's imagination.}

\begin{itemize}
\item Player 2 believes player 1 has the expected payoff $U^1 = 0.5 \times \cdots + 0.5 \times \cdots$, and player 1 knows what player 2 believes.
\item In the view of player 1, there is actually no expected payoff, since the which game to play is determined for him/her.
\end{itemize}


We calculate player 1's "expected payoff" according to player 2's belief for 2 reasons:

\begin{itemize}
\item Player 2 believes player 1 has this payoff, and acts according to it. This will in return affect palyer 1's behavior.
\item Even though player 1's "expected payoff" is calcualted by $U^1 = 0.5 \times \cdots + 0.5 \times \cdots$, what it reflects is actually his/her peference given which game is played. For example:

\begin{itemize}
\item We know if Game 1 is played and player 2 chooses L, then $D \succsim U$ for player 1.
\item In the Bayesian normal form, this preference is reflected by: if player 2 chooses L, $DU' \succsim UU'$, $DD' \succsim UD'$, i.e. no matter $U'$ or $D'$ is chosen in Game 2, once Game 1 is played and player 2 chooses L,  $D \succsim U$.
\end{itemize}
(Actually player 1's  preference can be preserved by any belief ($0<Pr<1$) of player 2. )

\end{itemize}

Anyway, the method is the same, simply calculate the expected payoff according to the belief for both of the two players.
\end{mdframed}



\subsection*{(b) For the Bayesian normal form found in part (a), determine a Nash equilibrium. Is there more than one Nash equilibrium?}

$(UD',R)$ is the unique Nash equilibrium.


%%%%%%%%%%%%%%%%%%%%%%%%%%%%%%%%%%%%%%%%%%%%%%%%%%%%%%%%%%%%%%%%%%%%%%%%%%%%%%%%%%%%%%%%%%%%%%
\section{Problem 2 - Bayesian normal form}

Consider the following two variants of a battle-of-the-sexes game. The game at the top --
variant (i) -- is of the usual kind where both players wish to meet each other, while the one
at the bottom -- variant (ii) -- has the unusual feature that $1$ wishes to avoid 2.\vspace{-6pt}

\begin{center}
Variant (i) \vspace{6pt}

$
\begin{array}{c|c|c|}
 & O & M \\
\hline
O & 3,1 & 0,0 \\
\hline
M & 0,0 & 1,3 \\
\hline
\end{array}
$
\end{center}

\begin{center}
Variant (ii) \vspace{6pt}

$
\begin{array}{c|c|c|}
 & O & M \\
\hline
O' & 0,1 & 3,0 \\
\hline
M' & 1,0 & 0,3 \\
\hline
\end{array}
$
\end{center}

\begin{mdframed}[backgroundcolor=blue!20,linecolor=white]
The payoff is 
\end{mdframed}


%
\subsection*{(a) For each of these games, determine the set of (pure) rationalizable strategies for each
player, and the set of pure-strategy and/or mixed-strategy Nash equilibria.}

All strategies are rationalizable. Nash equilibria: (i) $(O,O)$, $(M,M)$, $\left( \left( \tfrac34, \tfrac14 \right), \left( \tfrac14, \tfrac34 \right) \right)$. (ii) $\left( \left( \tfrac34, \tfrac14 \right), \left( \tfrac34, \tfrac14 \right) \right)$.


To derive the mixed-strategy Nash equilibrium in variant (i), let $p$ the probability of $1$ choosing $O$ and $q$ being the probability of $2$ choosing $O$. Payoff for $1$ choosing $O$ is: $3q$ and the payoff for $1$ choosing $M$ is $1-q$. They are equally good if $3q = 1-q$, which is equivalent to $q = \tfrac14$ (and $1-q=\tfrac34$). Payoff for $2$ choosing $O$ is: $p$ and the payoff for $2$ choosing $M$ is $3(1-p)$. They are equally good if $p = 3(1-p)$, which is equivalent to $p = \tfrac34$ (and $1-p=\tfrac14$).


To derive the mixed-strategy Nash equilibrium in variant (ii), let $p$ the probability of $1$ choosing $O'$ and $q$ being the probability of $2$ choosing $O$. Payoff for $1$ choosing $O'$ is: $3(1-q)$ and the payoff for $1$ choosing $M'$ is $q$. They are equally good if $3(1-q) = q$, which is equivalent to $q = \tfrac34$ (and $1-q=\tfrac14$). Payoff for $2$ choosing $O$ is: $p$ and the payoff for $2$ choosing $M$ is $3(1-p)$. They are equally good if $p = 3(1-p)$, which is equivalent to $p = \tfrac34$ (and $1-p=\tfrac14$).

%
\subsection*{(b) Assume next that only player $1$ knows which game is being played, while player $2$ thinks that
the two games are equally likely. Model this situation in an ex ante perspective by specifying
the Bayesian normal form.}

\begin{center}
$
\begin{array}{c|c|c|}
 & O & M \\
\hline
OO' & \tfrac32,1 & \tfrac32,0 \\
\hline
OM' & 2,\tfrac12 & 0,\tfrac32 \\
\hline
MO' & 0,\tfrac12 & 2,\tfrac32 \\
\hline
MM' & \tfrac12,0 & \tfrac12,3 \\
\hline
\end{array}
$
\end{center}
%
\subsection*{(c) For the Bayesian normal form found in part (b), determine the set of
(pure) rationalizable strategies for each player, and the set of pure-strategy and/or
mixed-strategy Nash equilibria. }

All strategies except $MM'$ are rationalizable.

Nash equilibria: $(MO', M)$, $\left( \left( \tfrac12, \tfrac12, 0, 0 \right), \left( \tfrac34, \tfrac14 \right) \right)$,  $\left( \left( \tfrac12, 0, \tfrac12, 0 \right), \left( \tfrac14, \tfrac34 \right) \right)$. 

\textit{To derive the mixed-strategy Nash equilibria, let $q$ being the probability of $2$ choosing $O$. Payoff for $1$ choosing $OO'$ is: $\tfrac32 q + \tfrac32 (1-q) = \tfrac32$. Payoff for $1$ choosing $OM'$ is: $2 q$. Payoff for $1$ choosing $MO'$ is: $2 (1-q)$. Payoff for $1$ choosing $MM'$ is: $\tfrac12 q + \tfrac12 (1-q) = \tfrac12$. Clearly, $MM'$ cannot be a best response.}

\textit{$OO'$ and $OM'$ are both best responses if $\tfrac32 = $2$ q$, which is equivalent to $q = \tfrac34$  (and $1-q=\tfrac14$). To show that $\left( \left( \tfrac12, \tfrac12, 0, 0 \right), \left( \tfrac34, \tfrac14 \right) \right)$ is a mixed-strategy Nash equilibrium, let $p$ be the probability of $1$ choosing $OO'$ and $1-p$ the probability of $1$ choosing $OM'$. Then payoff for $2$ choosing $O$ is $p + \tfrac12 (1-p)$ and the payoff for $2$ choosing $M$ is $\tfrac32 (1-p)$. They are equally good if $p + \tfrac12 (1-p) = \tfrac32 (1-p)$, which is equivalent to $p = \tfrac12$ (and $1-p = \tfrac12$).}

\textit{$OO'$ and $MO'$ are both best responses if $\tfrac32 = $2$ (1-q)$, which is equivalent to $q = \tfrac14$ (and $1-q=\tfrac34$). To show that $\left( \left( \tfrac12, 0, \tfrac12, 0 \right), \left( \tfrac14, \tfrac34 \right) \right)$ is a mixed-strategy Nash equilibrium, let $p'$ be the probability of $1$ choosing $OO'$ and $1-p'$ the probability of $1$ choosing $MO'$. Then payoff for $2$ choosing $O$ is $p' + \tfrac12 (1-p')$ and the payoff for $2$ choosing $M$ is $\tfrac32 (1-p')$. They are equally good if $p' + \tfrac12 (1-p') = \tfrac32 (1-p')$, which is equivalent to $p' = \tfrac12$ (and $1-p' = \tfrac12$).}
%


%%%%%%%%%%%%%%%%%%%%%%%%%%%%%%%%%%%%%%%%%%%%%%%%%%%%%%%%%%%%%%%%%%%%%%%%%%%%%%%%%%%%%%%%%%%%%%
\section{Problem 3 - Rationalizability in incomplete information games}

(Watson Exercise 26.3 pp.~469-470)
Suppose that nature selects A with probability $1>2$ and B with probability
$1>2$. If nature selects A, then players $1$ and $2$ interact according to matrix
``A.'' If nature selects B, then the players interact according to matrix ``B.''
These matrices are pictured here. Suppose that, before the players select
their actions, player $1$ observes nature's choice. That is, player $1$ knows
from which matrix the payoffs are drawn, and player $1$ can condition his
or her decision on this knowledge. Player $2$ does not know which matrix is
being played when he or she selects between L and R.

\subsection*{(a)} Draw the extensive-form representation of this game. Also represent
this game in Bayesian normal form. Compute the set of rationalizable
strategies and find the Nash equilibria.



\subsection*{(b)} Consider a three-player interpretation of this strategic setting in which each of player 1's types is modeled as a separate player. That is, the
game is played by players 1A, 1B, and 2. Assume that player 1A's
payoff is zero whenever nature chooses B; likewise, player 1B's payoff
is zero whenever nature selects A. Depict this version of the game in the
extensive form (remember that payoff vectors consist of three numbers)
and in the normal form. Compute the set of rationalizable strategies and
find the Nash equilibria.



\subsection*{(c)} Explain why the predictions of parts (a) and (b) are the same in regard
to equilibrium but different in regard to rationalizability. (Hint: The
answer has to do with the scope of the players' beliefs.)


%%%%%%%%%%%%%%%%%%%%%%%%%%%%%%%%%%%%%%%%%%%%%%%%%%%%%%%%%%%%%%%%%%%%%%%%%%%%%%%%%%%%%%%%%%%%%%
\section{Problem 4 - Differentiated duopoly}

(Watson Exercise 26.5 pp.~470.)

Consider a differentiated duopoly market in which firms compete by selecting
prices and produce to fill orders. Let p1 be the price chosen by firm
1 and let p2 be the price of firm 2. Let q1 and q2 denote the quantities demanded
(and produced) by the two firms. Suppose that the demand for firm
1 is given by $q_1 = 22 − 2p_1 + p_2$ , and the demand for firm $2$ is given by
$q_2 = 22 − 2p_2 + p_1$ . Firm $1$ produces at a constant marginal cost of 10 and
no fixed cost. Firm $2$ produces at a constant marginal cost of c and no fixed
cost. The payoffs are the firms' individual profits.

\subsection*{(a)} The firms' strategies are their prices. Represent the normal form by
writing the firms' payoff functions.
\subsection*{(b)} Calculate the firms' best-response functions.
\subsection*{(c)} Suppose that $c = 10$ so the firms are identical (the game is symmetric).
Calculate the Nash equilibrium prices.
\subsection*{(d)} Now suppose that firm $1$ does not know firm 2's marginal cost c. With
probability $1>2$ nature picks $c = 14$, and with probability 1>2 nature
picks $c = 6$. Firm $2$ knows its own cost (that is, it observes nature's
move), but firm $1$ only knows that firm 2's marginal cost is either 6 or
$14$ (with equal probabilities). Calculate the best-response functions of
player $1$ and the two types ($c = 6$ and $c = 14$) of player $2$ and calculate
the Bayesian Nash equilibrium quantities.

\end{document}
