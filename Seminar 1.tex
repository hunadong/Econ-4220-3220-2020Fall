\documentclass{article}
\usepackage[utf8]{inputenc}
\usepackage[T1]{fontenc}
\usepackage[english]{babel}

\usepackage{amssymb}
\usepackage{fourier}
\usepackage{microtype}

\title{Seminar 1}
\author{Xiaoguang Ling }
\date{August 26. 2020}

\begin{document}

\maketitle

\section{J.R. 1.8 Axioms of consumer choice}

Sketch a map of indifference sets that are all parallel, negatively sloped straight lines, with preference
increasing north-easterly.We know that preferences such as these satisfy Axioms 1, 2, 3, and 4. Prove
that they also satisfy Axiom 5. Prove that they do not satisfy Axiom 5.

\subsection{Axiom 1. Completeness}
\textbf{We can always choose}

$\forall \ x^1, x^2$ in $X$, we have: $x^1 \succsim  x^2$ or $x^2 \succsim  x^3$  or both

\subsection{Axiom 2. Transitivity}


\subsection{Axiom 3. Continuity}


\subsection{Axiom 4'. Local non-satiation}

\subsection{Axiom 4. Strict monotonicity}

\subsection{Axiom 5'. Convexity}

\subsection{Axiom 4. Strict convexity}

\section{J.R. 1.9}

1.9 Sketch a map of indifference sets that are all parallel right angles that ‘kink’ on the line $x_1 = x_2$. If
preference increases north-easterly, these preferences will satisfy Axioms 1, 2, 3, and 4. Prove that
they also satisfy Axiom 5. Do they satisfy Axiom 4? Do they satisfy Axiom 5?

\section{J.R. 1.13}
1.13 A consumer has lexicographic preferences over $x ∈ R2$ if the relation  satisfies $x_1, x_2$ whenever
$x_1^1 > x_1^2$, or $x_1^1 = x_1^2$ and $x_1^1 ≥ x_1^2$.

(a) Sketch an indifference map for these preferences.

(b) Can these preferences be represented by a continuous utility function? Why or why not?


\section{J.R. 1.15}
1.15 Prove that the budget set, B, is a compact, convex set whenever p  0.
\section{J.R. 1.26}
1.26 A consumer of two goods faces positive prices and has a positive income. His utility function is
$$u(x_1, x_2) = x_1$$ Derive the Marshallian demand functions.

\section{J.R. 1.27}
A consumer of two goods faces positive prices and has a positive income. His utility function is
$$u(x_1, x_2) = max[ax_1, ax_2] + min[x_1, x_2], where 0 < a < 1.$$
Derive the Marshallian demand functions.




\end{document}
