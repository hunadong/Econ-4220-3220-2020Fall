\documentclass{article}
\usepackage[utf8]{inputenc}
\usepackage[T1]{fontenc}
\usepackage[english]{babel}
\setlength{\parindent}{0pt}
\usepackage{hyperref}
\hypersetup{
    colorlinks=true,
    linkcolor=blue,
    filecolor=magenta,      
    urlcolor=cyan}
\usepackage{graphicx}
\graphicspath{ {./pic/} }
\usepackage{multicol}
\usepackage{lscape}

\usepackage{fourier,amssymb,microtype,amsmath,gensymb}
\newcommand{\R}{\mathbb{R}}
\usepackage{mdframed,caption,xcolor}
\usepackage{tikz,tkz-euclide}
\usepackage{multirow}
\title{Seminar 7. Static and dynamic games}
\author{Xiaoguang Ling \\  \href{xiaoguang.ling@econ.uio.no}{xiaoguang.ling@econ.uio.no}}
\date{\today}

\begin{document}

\maketitle

%%%%%%%%%%%%%%%%%%%%%%%%%%%%%%%%%%%%%%%%%%%%%%%%%%%%%%%%%%%%%%%%%%%%%%%%%%%%%%%%%%%%%%%%%%%%%%


%%%%%%%%%%%%%%%%%%%%%%%%%%%%%%%%%%%%%%%%%%%%%%%%%%%%%%%%%%%%%%%%%%%%%%%%%%%%%%%%%%%%%%%%%%%%%%
\section{Problem 1 - Simultaneous and sequential moves with complete information}


You and a friend are in a restaurant, and the owner offers both of you an 8-slice pizza under
the following condition. Each of you must simultaneously announce how many slices you would
like; that is, each player $i \in \{1, 2\}$ names his/her desired amount of pizza, $0 \leq s_i
\leq 8$. If $s_1 + s_2 \leq 8$, then the players get their demands (and the owner eats any
leftover slices). If $s_1 + s_2 > 8$, then the players get nothing. Assume that you each care
only about how much pizza you individually consume, preferring more pizza to less.


%***************************************************
\subsection{What is (are) each player's best response(s) for each of the possible demands for his/her
opponent?}


Best response set if opponent chooses $0$: $\{8\}$ \\
Best response set if opponent chooses $1$: $\{7\}$ \\
Best response set if opponent chooses $2$: $\{6\}$ \\
Best response set if opponent chooses $3$: $\{5\}$ \\
Best response set if opponent chooses $4$: $\{4\}$ \\
Best response set if opponent chooses $5$: $\{3\}$ \\
Best response set if opponent chooses $6$: $\{2\}$ \\
Best response set if opponent chooses $7$: $\{1\}$ \\
Best response set if opponent chooses $8$: $\{0,1,2,3,4,5,6,7,8\}$


\begin{table}
\setlength{\extrarowheight}{2pt}
\begin{tabular}{cc|c|c|c|c|c|c|c|c|c|}
  & \multicolumn{3}{c}{} & \multicolumn{4}{c}{Player $2$} & \multicolumn{3}{c}{} \\
  & \multicolumn{1}{c}{} & \multicolumn{1}{c}{$0$} & \multicolumn{1}{c}{$1$} & \multicolumn{1}{c}{$2$} 
  & \multicolumn{1}{c}{$3$} & \multicolumn{1}{c}{$4$} & \multicolumn{1}{c}{$5$} & \multicolumn{1}{c}{$6$} 
  & \multicolumn{1}{c}{$7$} & \multicolumn{1}{c}{$8$}  \\\cline{3-11}
            & $0$ & $(0,0)$ & $(0,1)$ & $(0,2)$ & $(0,3)$ & $(0,4)$ & $(0,5)$ & $(0,6)$ & $(0,7)$ & $(0,8)$ \\ \cline{3-11}  
            & $1$ & $(1,0)$ & $(1,1)$ & $(1,2)$ & $(1,3)$ & $(1,4)$ & $(1,5)$ & $(1,6)$ & $(1,7)$ & $(0,0)$ \\ \cline{3-11}
  			& $2$ & $(2,0)$ & $(2,1)$ & $(2,2)$ & $(2,3)$ & $(2,4)$ & $(2,5)$ & $(2,6)$ & $(0,0)$ & $(0,0)$\\\cline{3-11}
            & $3$ & $(3,0)$ & $(3,1)$ & $(3,2)$ & $(3,3)$ & $(3,4)$ & $(3,5)$ & $(0,0)$ & $(0,0)$ & $(0,0)$ \\\cline{3-11}
Player $1$  & $4$ & $(4,0)$ & $(4,1)$ & $(4,2)$ & $(4,3)$ & $(4,4)$ & $(0,0)$ & $(0,0)$ & $(0,0)$ & $(0,0)$ \\\cline{3-11}
            & $5$ & $(5,0)$ & $(5,1)$ & $(5,2)$ & $(5,3)$ & $(0,0)$ & $(0,0)$ & $(0,0)$ & $(0,0)$ & $(0,0)$ \\\cline{3-11}
            & $6$ & $(6,0)$ & $(6,1)$ & $(6,2)$ & $(0,0)$ & $(0,0)$ & $(0,0)$ & $(0,0)$ & $(0,0)$ & $(0,0)$ \\\cline{3-11}
            & $7$ & $(7,0)$ & $(7,1)$ & $(0,0)$ & $(0,0)$ & $(0,0)$ & $(0,0)$ & $(0,0)$ & $(0,0)$ & $(0,0)$ \\\cline{3-11}
            & $8$ & $(8,0)$ & $(0,0)$ & $(0,0)$ & $(0,0)$ & $(0,0)$ & $(0,0)$ & $(0,0)$ & $(0,0)$ & $(0,0)$ \\\cline{3-11}

\end{tabular}
\end{table}



%
%***************************************************
\subsection{Find all the pure-strategy Nash equilibria}

$(0,8),(1,7),(2,6),(3,5),(4,4),(5,3),(6,2),(7,1),(8,0),(8,8)$
%

Reconsider the situation above, but assume now that player 1 makes her demand before
player 2 makes his demand. Player 2 observes player 1's demand before making his choice.
%
%
%***************************************************
\subsection{Explain what a strategy is for player 2 in this game with sequential moves.}

Determines a choice for player 2 for each possible choice for player 1. Player 2 has $9^9$ strategies.
%
%***************************************************
\subsection{Find all the pure-strategy Nash equilibrium outcomes. }

\textit{(1) $s^\ast_1 \in \{0,1,2,3,4,5,6,7,8\}$ and \\
$s^\ast_2(s_1) = 8 - s_1$ if $s_1 = s^\ast_1$, \\
$s^\ast_2(s_1) > 8 - s_1$ if $s_1 > s^\ast_1$, \\
$s^\ast_2(s_1) \in \{0,1,2,3,4,5,6,7,8\}$ if $s_1 < s^\ast_1$. \\
\textit{Example:} $(4, (8,7,6,5,4,4,4,4,4))$\textit{. Here player 2 demands the pieces that are left if player 1 does not demand more than 4 pieces, but demands 4 pieces if player 1 demands more than 4 pieces. To demand 4 pieces is a best response for player 1, given that he will not get anything if demands more than 4 pieces. It is a best response for player 2, given that player 1 demands 4 pieces, as his strategy specifies.} \\
(2) $s^\ast_1 = 8$ and \\
$s^\ast_2(s_1) > 8 - s_1$ if $s_1 \in \{1,2,3,4,5,6,7,8\}$, \\
$s^\ast_2(s_1) \in \{0,1,2,3,4,5,6,7,8\}$ if $s_1 = 0$.} \\
\textit{Eksempel:} $(8, (8,8,8,8,8,8,8,8,8))$\textit{. Here player 2 demands all the 8 pieces independently of what player 1 demands. To demand all the 8 pieces is a best response for player 1, given that he will not get anything anyway. It is a best response for player 2, given that player 1 demands all the 8 pieces, as his strategy specifies.}

%
%***************************************************
\subsection{Find all the pure-strategy subgame perfect equilibria.}

\textit{(1) $s^\ast_1 = 7$ and \\
$s^\ast_2(s_1) = 8 - s_1$ if $s_1 \in \{0,1,2,3,4,5,6,7\}$, \\
$s^\ast_2(s_1) > 8 - s_1$ if $s_1 = 8$. \\
\textit{Example:} $(7, (8,7,6,5,4,3,2,1,1))$\textit{. Here player 2 demands the pieces that are left if player 1 demands less that all the 8 pieces, but demands 1 piece if player 1 demands all 8 pieces. This is a best response for player 2, not only if player 1 demands 7 pieces, as his strategy specifies, but also for all other choices that player 1 might do. To demand 7 pieces is a best response for player 1, given that he will not get anything if he demands all the 8 pieces.}
\\
(2) $s^\ast_1 = 8$ and \\
$s^\ast_2(s_1) = 8 - s_1$ if $s_1 \in \{0,1,2,3,4,5,6,7,8\}$.} \\
\textit{That is:} $(8, (8,7,6,5,4,3,2,1,0))$\textit{. Here player 2 requires the pieces that are left. This is a best response for player 2, not only if player 1 demands all the 8 pieces, as his strategy specifies, but also for all other choices that player 1 might do. To demand all the 8 pieces is a best response for player 1.}
%


\vspace{-6pt}

\bigskip

%%%%%%%%%%%%%%%%%%%%%%%%%%%%%%%%%%%%%%%%%%%%%%%%%%%%%%%%%%%%%%%%%%%%%%%%%%%%%%%%%%%%%%%%%%%%%%
\section{Problem 2 - Best response sets} 

 Exercise 6.4 

For the game of Figure 6.2 (Watson pp.55), determine the following best-response sets.


{\includegraphics[width=0.8\textwidth]{7.f6_2}
\label{fig:f6_2}}
\vspace{2mm}



%***************************************************
\subsection{$BR_1(\theta_2)$ for $\theta_2 = (1/6,1/3,1/2)$}

%***************************************************
\subsection{$BR_2(\theta_1)$ for $\theta_1 = (1/6,1/3,1/2)$}

%***************************************************
\subsection{$BR_1(\theta_2)$ for $\theta_2 = (1/4,1/8,5/8)$}

%***************************************************
\subsection{$BR_1(\theta_2)$ for $\theta_2 = (1/3,1/3,1/3)$}

%***************************************************
\subsection{$BR_2(\theta_1)$ for $\theta_1 = (1/2,1/2,0)$}



\textit{\indent (a) $\{U\}$ \\ \indent (b) $\{R\}$ \\ \indent (c) $\{U\}$ \\ \indent (d) $\{U,D\}$ \\ \indent (e) $\{L,R\}$}

\bigskip

\section{Problem 3} \textit{(Best response functions, Nash equilibria, rationalizable strategies)}

Watson Exercise 9.6 

Consider a game in which, simultaneously, player 1 selects any real number
$x$ and player 2 selects any real number $y$. The payoffs are given by:
$$u_1(x,y)= 2x-x^2+2xy$$
$$u_2(x,y)= 10y-2xy -y^2$$

%***************************************************
\subsection{Calculate and graph each player’s best-response function as a function
of the opposing player’s pure strategy.}
(a) $BR_1(y) = 1+y$, $BR_2(x) = 5-x$.
%***************************************************
\subsection{Find and report the Nash equilibria of the game.}
(b) $(3,2)$.
%***************************************************
\subsection{Determine the rationalizable strategy profiles for this game.}

For each player the set of rationalizable strategies equals $(-\infty, \infty)$.

\bigskip

\section{Problem 4 - True or False?}

For each of the statements, if true, try to explain why, and if false, provide a
counter-example.
%
%
\begin{itemize}
%
\item[(a)] In a finite extensive-form game of perfect information, there always exists a subgame perfect
Nash equilibrium. 

 True. A subgame-perfect Nash equilibrium can be constructed by using backward induction.
%
\item[(b)] In a finite extensive-form game of perfect information, there always exists a unique subgame perfect Nash equilibrium. 

False. Let player 1 choose `out', leading to the payoff vector $(1,1)$ or `in', whereafter player 2 can choose `a' leading to the payoff vector $(2,2)$, or `b' leading to the payoff vector $(0,2)$. Verify that there are two subgame-perfect Nash equilibria depending on what player 2 does if player 1 chooses `in'.
%
\end{itemize}
\vspace{-6pt}

\bigskip

\section{Problem 5- Firm-union bargaining}

A firm's output is $L(100-L)$ when it uses $L \leq 50$ units of labor, and $2500$ when it uses
$L \geq 50$ units of labor. The price of output is $1$. A union that represents workers
presents a wage demand (a nonnegative number $w$), which the firm either accepts or rejects. If
the firm accepts the demand, it chooses the number $L$ of workers to employ (which you should
take to be a continuous variable, not an integer); if it rejects the demand, no production
takes place ($L = 0$). The firm's preferences are represented by its profits, the union's
preferences are represented by  the value of $wL$.

%***************************************************
\subsection{Formulate this situation as an extensive game with perfect information. }

Players: $N = \{U, F \}$. \\ Strategies: $U$ chooses a wage $w$ from the set of non-negative number; $F$ chooses a function that to any non-negative wage $w$ determines a non-negative employment $L(w)$. \\ Payoffs: The union's payoff is $wL(w)$; the firm's payoff is $L(w)(100-L(w)) - wL(w)$. \\ No need to have a specific accept/reject decision at $L(w) = 0$ is in effect a rejection by the firm of the demand $w$, giving both a payoff of $0$.
%
%***************************************************
\subsection{Find the subgame perfect equilibrium (equilibria?) of the game.}

Maximizing the firm's payoff yields $L(w) = \tfrac{100-w}2$ for $w \leq 100$ and $L(w) = 0$ otherwise. \\ The union's best response to this strategy is setting $w=50$. 
%
%***************************************************
\subsection{Is there an outcome of the game that both parties prefer to any subgame perfect
equilibrium outcome? }

The subgame-perfect equilibrium outcome is $w=50$ and $L=25$, yielding a payoff of $1250$ for the union and $625$ for the firm. Joint surplus is maximized for $L = 50$, yielding a maximized total surplus of $2500$. At this employment level, any wage $w$ between $25$ and $37.5$ would lead to a Pareto-improvement. 
%
%***************************************************
\subsection{Find a Nash equilibrium for which the outcome differs from any subgame perfect
equilbrium outcome.}

Consider $L(w) = \tfrac{100-w}2$ for $w \leq 20$ and $L(w) = 0$ otherwise. Then the union's best response is $w=20$, leading to the employment $L=40$ and the payoffs $800$ for the union and $1600$ for the firm. This is a Nash equilibrium, but it is not subgame-perfect. 
%





\end{document}