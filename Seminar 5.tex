\documentclass{article}
\usepackage[utf8]{inputenc}
\usepackage[T1]{fontenc}
\usepackage[english]{babel}
\setlength{\parindent}{0pt}
\usepackage{hyperref}
\hypersetup{
    colorlinks=true,
    linkcolor=blue,
    filecolor=magenta,      
    urlcolor=cyan}
\usepackage{graphicx}
\graphicspath{ {./pic/} }
\usepackage{multicol}
\usepackage{lscape}

\usepackage{fourier,amssymb,microtype,amsmath,gensymb}
\newcommand{\R}{\mathbb{R}}
\usepackage{mdframed,caption,xcolor}
\usepackage{tikz,tkz-euclide}

\title{Seminar 5. Production Theory}
\author{Xiaoguang Ling \\  \href{xiaoguang.ling@econ.uio.no}{xiaoguang.ling@econ.uio.no}}
\date{\today}

\begin{document}

\maketitle

%%%%%%%%%%%%%%%%%%%%%%%%%%%%%%%%%%%%%%%%%%%%%%%%%%%%%%%%%%%%%%%%%%%%%%%%%%%%%%%%%%%%%%%%%%%%%%

%%%%%%%%%%%%%%%%%%%%%%%%%%%%%%%%%%%%%%%%%%%%%%%%%%%%%%%%%%%%%%%%%%%%%%%%%%%%%%%%%%%%%%%%%%%%%%
\section{Jehle \& Reny 3.35}

Calculate the \textbf{cost function} and the \textbf{conditional input demands} for the linear production function,
$y = \Sigma^n_{i=1} \alpha_1 x_i$.

\begin{mdframed}[backgroundcolor=blue!20,linecolor=white]


\textbf{ASSUMPTION 3.1 Properties of the Production Function} (Jehle \& Reny pp.127)

The production function, $f : \R^n_+ \to \R_+$ , is continuous, strictly increasing, and strictly
quasiconcave on $\R^n_+$, and $f (0) = 0$.



\textbf{DEFINITION 3.5 The Cost Function} (Jehle \& Reny pp.136)

The cost function, defined for all input prices $w \gg 0$ and all output levels $y \in f(\R^n_+)$ is the
minimum-value function,
$$c(w,y) \equiv \min_{x \in \R^n_+} w \cdot x , \ \ s.t. \ \ f(x) \ge y.$$
The solution $x(w, y)$ is referred to as the firm’s \textbf{conditional input demand}, because it is conditional on the level of output $y$.

\begin{itemize}
\item  \textbf{Conditional input demand} is similar to Hicksian demands for consumers
\item  The difference is that cost minimization may not lead to profit maximization.
\end{itemize}
\end{mdframed}


%%%%%%%%%%%%%%%%%%%%%%%%%%%%%%%%%%%%%%%%%%%%%%%%%%%%%%%%%%%%%%%%%%%%%%%%%%%%%%%%%%%%%%%%%%%%%%
\section{Jehle \& Reny 3.46}

\begin{itemize}
\item Verify Theorem 3.7 for the profit function obtained in Example 3.5. 
\item Verify Theorem 3.8 for the associated output supply and input demand functions.
\end{itemize}

\begin{mdframed}[backgroundcolor=blue!20,linecolor=white]

\textbf{DEFINITION 3.7 The Profit Function} (Jehle \& Reny pp.148)

The firm’s profit function depends only on input and output prices and is defined as the
maximum-value function,




\textbf{THEOREM 3.7 Properties of the Profit Function} (Jehle \& Reny pp.148)


\end{mdframed}






\begin{mdframed}[backgroundcolor=blue!20,linecolor=white]

\textbf{THEOREM 3.8 Properties of Output Supply and Input Demand Functions} (Jehle \& Reny pp.149)

\end{mdframed}


%%%%%%%%%%%%%%%%%%%%%%%%%%%%%%%%%%%%%%%%%%%%%%%%%%%%%%%%%%%%%%%%%%%%%%%%%%%%%%%%%%%%%%%%%%%%%%
\section{Jehle \& Reny 3.49}

\begin{enumerate}
\item Derive the \textbf{cost function} for the production function in Example 3.5. 

\item Solve $\max_y py - c(w, y)$ 

\item Compare its solution, $y(p,w)$, to the solution in (E.5). Check that $\pi (p,w) = py(p,w) - c(w, y(p,w))$. 

\item Supposing that $\beta> 1$, confirm our conclusion that profits are minimised when the
first-order conditions are satisfied by showing that marginal cost is decreasing at the solution. 

\item Sketch your results.

\end{enumerate}















\end{document}
