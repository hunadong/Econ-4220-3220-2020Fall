\documentclass{article}
\usepackage[utf8]{inputenc}
\usepackage[T1]{fontenc}
\usepackage[english]{babel}
\setlength{\parindent}{0pt}
\usepackage{hyperref}
\hypersetup{
    colorlinks=true,
    linkcolor=blue,
    filecolor=magenta,      
    urlcolor=cyan}
\usepackage{graphicx}
\graphicspath{ {./pic/} }
\usepackage{multicol}
\usepackage{lscape}

\usepackage{fourier,amssymb,microtype,amsmath,gensymb}
\newcommand{\R}{\mathbb{R}}
\usepackage{mdframed,caption,xcolor}
\usepackage{tikz,tkz-euclide}

\title{Seminar 5. Production Theory}
\author{Xiaoguang Ling \\  \href{xiaoguang.ling@econ.uio.no}{xiaoguang.ling@econ.uio.no}}
\date{\today}

\begin{document}

\maketitle

%%%%%%%%%%%%%%%%%%%%%%%%%%%%%%%%%%%%%%%%%%%%%%%%%%%%%%%%%%%%%%%%%%%%%%%%%%%%%%%%%%%%%%%%%%%%%%
\begin{landscape}
\section*{Production Duality}


\vspace{3mm}
{\scriptsize
\usetikzlibrary{arrows.meta}
\tikzset{%
  >={Latex[width=1mm,length=1mm]},
  % Specifications for style of nodes:
            base/.style = {rectangle, rounded corners, draw=black,
                           minimum width=2cm, minimum height=1cm,
                           text centered, font=\sffamily},
  Starts/.style = {base, fill=blue!30},
         process/.style = {base, minimum width=2.5cm, fill=orange!15,
                           font=\ttfamily},
}
\begin{tikzpicture}[node distance=2cm,
every node/.style={fill=white, font=\sffamily}, align=center]
\node (start) [Starts] {Profit Maximization: \\ $\max_{(x,y) \ge 0} \ py - wx, s.t.  f(x) \ge y$};
\node (1) [process, below of=start] {Output Supply Function: \\ $y^* \equiv y(p,w)$ \\  \\Input Demand Functions: \\ $x^* \equiv x(p,w)$};
\node (2) [process, below of=1] {Profit Function: \\ $\pi (p,w) =  py^* - wx^*$};

\node (3) [process, left of=2,xshift=-1.5cm,yshift=1cm]  {Hotelling's lemma (pp.148): \\ $y_(p,w) = \frac{\partial \pi(p,w)}{\partial p}$ \\  \\   $x_i(p,w) = - \frac{\partial \pi(p,w)}{\partial w_i}$};


\draw[->]  (start) -- (1);
\draw[->] (1) -- (2);
\draw[dashed]  (2) --  (3);
\draw[dashed]  (3) --  (1);


{\tiny
\path (start) to  node {solution $x^*, y^*$} (1); 
\path (1) to  node {maximized value} (2); 
}
  
%%%%%%%%%%%%%%%%%%%%%%%%%%%%%%%%%%%%%%%%%%%%%%%%%%%%%%%%%%%%%%%%%%%%%%%%%%%%%%%%%%%%%%%%%%%%%%%%%%%%%%%%%%%%%

\node (start2) [Starts, right of = 1, xshift =6cm, yshift=2cm] {Cost Minimization: \\ $\min_{x \in \R^n_+} \ w x, s.t. f(x) \ge y$};
\node (21) [process, below of=start2] {Conditional Input Demand: \\ $x(w,y)$};
\node (22) [process, below of=21] {Cost Function: \\ $c(w,y) = w x(w,y)$};
\node (23) [process, right of=22,xshift=1.5cm,yshift=1cm]  {Shephard's lemma(pp.37): \\ $\frac{\partial c(w^0,y^0)}{\partial w_i} = x_i(w^0,y^0)$};
\node (24) [process, below of=22] {Production Function (pp.144): \\ $f(x) \equiv max\{y \ge 0 | wx \ge c(w,y), \forall w \gg 0 \}$};
\draw[->]  (start2) -- (21);
\draw[->]  (21) -- (22);
\draw[dashed]  (22) --  (23);
\draw[dashed]  (23) --  (21);
\draw[->] (22) --  (24);
{\tiny
\path (start2) to  node {solution $x^*$} (21); 
\path (21) to  node {minimized value} (22); 
}
%%%%%%%%%%%%%%%%%%%%%%%%%%%%%%%%%%%%%%%%%%%%%%%%%%%%%%%%%%%%%%%%%%%%%%%%%%%%%%%%%%%%%%%%%%%%%%%%%%%%%%%%%%%%
\node (31) [process, right of=1,xshift=2cm,yshift=02cm] {FOC(pp.146): \\ $\frac{\partial f(x^*) / \partial x_i}{\partial f(x^*) / \partial x_j} = \frac{w_i}{w_j}$};

\draw[dashed]  (start) --  (31);
\draw[dashed]  (31) --  (start2);

\end{tikzpicture}
}
\end{landscape}

%%%%%%%%%%%%%%%%%%%%%%%%%%%%%%%%%%%%%%%%%%%%%%%%%%%%%%%%%%%%%%%%%%%%%%%%%%%%%%%%%%%%%%%%%%%%%%

\section{Jehle \& Reny 3.35}

Calculate the \textbf{cost function} and the \textbf{conditional input demands} for the linear production function,
$y = \Sigma^n_{i=1} \alpha_1 x_i$.

\begin{mdframed}[backgroundcolor=blue!20,linecolor=white]

\textbf{Production Function}(Jehle \& Reny pp.127)

We use a function $y = f(x)$ to denote $y$ units of a certain commodity is produced using input $x$, where $x \in \R^n_+, y \in \R^1_+$

\vspace{2mm}

\textbf{ASSUMPTION 3.1 Properties of the Production Function} (Jehle \& Reny pp.127)

The production function, $f : \R^n_+ \to \R_+$ , is continuous, strictly increasing, and strictly
quasiconcave on $\R^n_+$, and $f (0) = 0$.

\vspace{2mm}

\textbf{DEFINITION 3.5 The Cost Function} (Jehle \& Reny pp.136)

The cost function, defined for all input prices $w \gg 0$ and all output levels $y \in f(\R^n_+)$ is the
minimum-value function,
$$c(w,y) \equiv \min_{x \in \R^n_+} w \cdot x , \ \ s.t. \ \ f(x) \ge y.$$
The solution $x(w, y)$ is referred to as the firm’s \textbf{conditional input demand}, because it is conditional on the level of output $y$.

\begin{itemize}
\item  \textbf{Conditional input demand} is similar to Hicksian demands for consumers
\item  The difference is that cost minimization may not lead to profit maximization.
\end{itemize}
\end{mdframed}


%%%%%%%%%%%%%%%%%%%%%%%%%%%%%%%%%%%%%%%%%%%%%%%%%%%%%%%%%%%%%%%%%%%%%%%%%%%%%%%%%%%%%%%%%%%%%%
\section{Jehle \& Reny 3.46}

\begin{itemize}
\item Verify Theorem 3.7 for the profit function obtained in Example 3.5. 
\item Verify Theorem 3.8 for the associated output supply and input demand functions.
\end{itemize}

\begin{mdframed}[backgroundcolor=blue!20,linecolor=white]

\textbf{DEFINITION 3.7 The Profit Function} (Jehle \& Reny pp.148)

The firm’s profit function depends only on input and output prices and is defined as the
maximum-value function,




\textbf{THEOREM 3.7 Properties of the Profit Function} (Jehle \& Reny pp.148)


\end{mdframed}






\begin{mdframed}[backgroundcolor=blue!20,linecolor=white]

\textbf{THEOREM 3.8 Properties of Output Supply and Input Demand Functions} (Jehle \& Reny pp.149)

\end{mdframed}


%%%%%%%%%%%%%%%%%%%%%%%%%%%%%%%%%%%%%%%%%%%%%%%%%%%%%%%%%%%%%%%%%%%%%%%%%%%%%%%%%%%%%%%%%%%%%%
\section{Jehle \& Reny 3.49}

\begin{enumerate}
\item Derive the \textbf{cost function} for the production function in Example 3.5. 

\item Solve $\max_y py - c(w, y)$ 

\item Compare its solution, $y(p,w)$, to the solution in (E.5). Check that $\pi (p,w) = py(p,w) - c(w, y(p,w))$. 

\item Supposing that $\beta> 1$, confirm our conclusion that profits are minimised when the
first-order conditions are satisfied by showing that marginal cost is decreasing at the solution. 

\item Sketch your results.

\end{enumerate}















\end{document}
