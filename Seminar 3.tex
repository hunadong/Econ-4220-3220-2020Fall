\documentclass{article}
\usepackage[utf8]{inputenc}
\usepackage[T1]{fontenc}
\usepackage[english]{babel}
\setlength{\parindent}{0pt}
\usepackage{hyperref}
\hypersetup{
    colorlinks=true,
    linkcolor=blue,
    filecolor=magenta,      
    urlcolor=cyan}
\usepackage{graphicx}
\graphicspath{ {./pic/} }
\usepackage{multicol}
\usepackage{lscape}

\usepackage{fourier,amssymb,microtype,amsmath,gensymb}
\newcommand{\R}{\mathbb{R}}
\usepackage{mdframed,caption,xcolor}
\usepackage{tikz,tkz-euclide}

\title{Seminar 3.Duality of Consumers Behavior}
\author{Xiaoguang Ling \\  \href{xiaoguang.ling@econ.uio.no}{xiaoguang.ling@econ.uio.no}}
\date{\today}

\begin{document}

\maketitle

%%%%%%%%%%%%%%%%%%%%%%%%%%%%%%%%%%%%%%%%%%%%%%%%%%%%%%%%%%%%%%%%%%%%%%%%%%%%%%%%%%%%%%%%%%%%%%

\begin{landscape}
\section*{Consumption Duality}

\begin{mdframed}[backgroundcolor=yellow!20,linecolor=white]

You will never lose your way with this Consumption Duality map!

All "derive this from that and verify some guy's equation"-like questions can be
solved by finding the correct (shortest) route.
\end{mdframed}
\vspace{3mm}
{\scriptsize
\usetikzlibrary{arrows.meta}
\tikzset{%
  >={Latex[width=1mm,length=1mm]},
  % Specifications for style of nodes:
            base/.style = {rectangle, rounded corners, draw=black,
                           minimum width=2cm, minimum height=1cm,
                           text centered, font=\sffamily},
  Starts/.style = {base, fill=blue!30},
         process/.style = {base, minimum width=2.5cm, fill=orange!15,
                           font=\ttfamily},
}
\begin{tikzpicture}[node distance=2cm,
    every node/.style={fill=white, font=\sffamily}, align=center]
  \node (start) [Starts] {Utility Maximization: \\ $\max_{x \in \R^n_+} \ u(x), s.t.  px \le y$};
  \node (1) [process, below of=start] {Marshallian Demand: \\ \ $x(p,y)$};
  \node (2) [process, below of=1] {Indirect Utility: \\ $v(p,y) = u[x(p,y)]$};
  \node (3) [process, left of=2,xshift=-1.5cm,yshift=1cm]  {Roy's Identity(pp.29): \\ $x_i(p^0,y^0) = - \frac{\partial v(p^0,y^0) / \partial p_i}{\partial v(p^0,y^0) / \partial y}$};
  \node (4) [process, below of=2] {Direct Utility (pp.82): \\ $u(x) = \min_{p \in \R^n_{++}}  v(p,1), s.t. px=1$};
  \node (5) [process, below of=4] {Inverse Demands (pp.84): \\ (Hotelling, Wold) \\ $p_i(x) = \frac{\partial u(x) / \partial x_i}{\Sigma^{n}_{j=1} x_j( \partial u(x) / \partial x_j)} $};
  \draw[->]  (start) -- (1);
  \draw[->] (1) -- (2);
  \draw[dashed]  (2) --  (3);
  \draw[dashed]  (3) --  (1);
  \draw[->] (2) --  (4);
  \draw[->] (4) --  (5);
{\tiny
\path (start) to  node {solution $x^*$} (1); 
\path (1) to  node {maximized value} (2); 
\path (4) to  node {solution $p^*$} (5); 
}
  
%%%%%%%%%%%%%%%%%%%%%%%%%%%%%%%%%%%%%%%%%%%%%%%%%%%%%%%%%%%%%%%%%%%%%%%%%%%%%%%%%%%%%%%%%%%%%%%%%%%%%%%%%%%%%

  \node (start2) [Starts, right of = 1, xshift =9cm, yshift=2cm] {Expenditure Minimization: \\ $\min_{x \in \R^n_+} \ px, s.t. u \ge \bar{u}$};
  \node (21) [process, below of=start2] {Hicksian Demand: \\ $x^h(p,\bar{u})$};
  \node (22) [process, below of=21] {Expenditure: \\ $e(p,\bar{u}) = px^h(p,\bar{u})$};
  \node (23) [process, right of=22,xshift=1.5cm,yshift=1cm]  {Shephard's lemma(pp.37): \\ $\frac{\partial e(p^0,u^0)}{\partial p_i} = x^h(p^0,u^0)$};
  \node (24) [process, below of=22] {Direct Utility (pp.75): \\ $u(x) \equiv max\{u \le 0 | x \in A(u) \}$};
  \draw[->]  (start2) -- (21);
  \draw[->]  (21) -- (22);
  \draw[dashed]  (22) --  (23);
  \draw[dashed]  (23) --  (21);
  \draw[->] (22) --  (24);
{\tiny
\path (start2) to  node {solution $x^*$} (21); 
\path (21) to  node {minimized value} (22); 
}
%%%%%%%%%%%%%%%%%%%%%%%%%%%%%%%%%%%%%%%%%%%%%%%%%%%%%%%%%%%%%%%%%%%%%%%%%%%%%%%%%%%%%%%%%%%%%%%%%%%%%%%%%%%%
  \node (31) [process, right of=1,xshift=3.5cm,yshift=2cm] {Slutsky equation(pp.53): \\ $\frac{\partial x_i(p,y)}{\partial p_j} = \frac{\partial x_i^h(p,u^*)}{\partial p_j} - x_j(p,y)\frac{\partial x_i(p,y)}{\partial y}$};
  \node (32) [process, below of=31,yshift=-4cm] {$e[p,v(p,y)] = y$ \\ $v[p,e(p,u)] = u$};
  \draw[dashed]  (1) --  (31);
  \draw[dashed]  (31) --  (21);
  \draw[<-]  (2) --  (32);
  \draw[->]  (32) --  (22);
  \draw[-,blue]   (21) -- (2.north east);
  \draw[-,red]   (22.north west) -- (1);
  
%\path (22.west) to  node {ddd} (21); 
 \draw [->,red,yshift=2cm] (22.north west) -- (21) ;
 \draw [->,blue,yshift=2cm] (2.north east) -- (1) ;

  \end{tikzpicture}
}
\end{landscape}

%%%%%%%%%%%%%%%%%%%%%%%%%%%%%%%%%%%%%%%%%%%%%%%%%%%%%%%%%%%%%%%%%%%%%%%%%%%%%%%%%%%%%%%%%%%%%%
\section{Jehle \& Reny 2.3}
Derive the consumer’s direct utility function if his indirect utility function has the form $v(p, y) =
yp_1^\alpha p_2^\beta$ for negative $\alpha$ and $\beta$.

\begin{mdframed}[backgroundcolor=blue!20,linecolor=white]
\textbf{THEOREM 2.3 Duality Between Direct and Indirect Utility}(Jehle \& Reny pp.81 )


Suppose that $u(x)$ is quasiconcave and differentiable on $\R^n_{++}$ with strictly positive partial derivatives there. Then for all $x \in \R^n_{++}$ , $v(p, p \cdot x)$, the indirect utility function generated
by $u(x)$, achieves a minimum in $p$ on $\R^n_{++}$, and

$$u(x) = \min_{p \in \R^n_{++}}  v(p,y), s.t. px=y$$

Let's call the solution $p^*$

Note that by \textbf{Theorem 1.6}(Jehle \& Reny pp.29), $v(p, y)$ is homogeneous of degree zero in $(p, y)$. We have $v(p, p \cdot x) = v(p/(p \cdot x), 1)$ whenever $p \cdot x > 0$. Thus the equation above can also be written as:

$$u(x) = \min_{p \in \R^n_{++}}  v(p,1), s.t. px=1$$

The solution $\hat{p} = p^* / p^* \cdot x = p^* / y$. We don't care about the difference between $\hat{p}$ and $p^*$ because once you substitute them into $v(p,p\cdot x)$, you have the same result (homogeneity of degree zero).
\end{mdframed}

$$u(x) = \min_{p \in \R^n_{++}}  v(p, 1) =
p_1^\alpha p_2^\beta, s.t. px=1$$

Lagrangian:

$$L = p_1^\alpha p_2^\beta + \lambda (1- p_1x_1 - p_2x_2)$$

Note there should not be interior solution since

\begin{itemize}
\item $\frac{\partial v(p_1,p_2,1)}{\partial p_1} = \alpha p_1^{\alpha -1}p_2^\beta, \alpha, \beta < 0. \lim_{p_1 \to 0} \frac{\partial v(p_1,p_2,1)}{\partial p_1} = - \infty$

\item $\frac{\partial v(p_1,p_2,1)}{\partial p_2} = p_1^{\alpha} \beta p_2^{\beta - 1}, \alpha, \beta < 0. \lim_{p_2 \to 0} \frac{\partial v(p_1,p_2,1)}{\partial p_2} = - \infty$

\item $v(p,1)$ is decreasing in $p$(this is always true for indirect utility function, see pp.29). For any $px<1$, you can always decrease $v(p, 1)$ by increasing $p$ until $px = 1$.
\end{itemize}

FOCs.

\begin{equation}
    \begin{cases}
\frac{\partial L}{\partial p_1} = \alpha p_1^{\alpha - 1} p_2^\beta - \lambda x_1= 0 \\
\frac{\partial L}{\partial p_2} = p_1^{\alpha} \beta p_2^{\beta - 1} - \lambda x_2 = 0 \\
p_1x_1 + p_2x_2 = 1
    \end{cases}
    \nonumber
\end{equation}

Simplify:

\begin{equation}
    \begin{cases}
 \alpha p_1^{\alpha - 1} p_2^\beta = \lambda x_1 \\
 \beta  p_1^{\alpha}  p_2^{\beta - 1} = \lambda x_2  \\
p_1x_1 + p_2x_2 = 1
    \end{cases}
    \label{eq:2_3_foc}   
\end{equation}

Take the ratio between first and second condition to get:

$$\frac{x_1}{x_2} = \frac{\alpha}{\beta} \frac{p_2}{p_1}$$

Thus: $p_2 = \frac{\beta}{\alpha}\frac{x_1}{x_2} p_1 $

Substitute $p_2$ with $p_1$ in the 3rd condition to get:

\begin{align*}
p_1x_1 + \frac{\beta}{\alpha}\frac{x_1}{x_2} p_1  x_2 &= 1 \\
p_1(x_1 + \frac{\beta}{\alpha} x_1) &= 1 \\
p_1^* &= \frac{1}{x_1(1+ \frac{\beta}{\alpha})} \\
\Rightarrow p_2^* &= \frac{\beta}{\alpha} \frac{x_1}{x_2} p_1 =\frac{\beta}{\alpha} \frac{x_1}{x_2}\frac{1}{x_1(1+ \frac{\beta}{\alpha})} = \frac{1}{x_2(1+ \frac{\alpha}{\beta})}
\end{align*}

Substitute $p^*_1$ and $p^*_2$ into $v(p,1)$ we get the minimized value, i.e. the direct utility function:

\begin{align*}
u(x_1.x_2) &= [\frac{1}{x_1(1+ \frac{\beta}{\alpha})}]^\alpha[\frac{1}{x_2(1+ \frac{\alpha}{\beta})}]^\beta \\
 &= A x_1^a x_2^b
\end{align*}

Where $A = [\frac{1}{1+ \frac{\beta}{\alpha}}]^\alpha[\frac{1}{1+ \frac{\alpha}{\beta}}]^\beta$ ,$a = -\alpha >0$ , $b= -\beta > 0$. The utility function is a Cobb-Douglas function.


\begin{mdframed}[backgroundcolor=yellow!20,linecolor=white]
As a cautious proof, you may want to check if $u(x)$ is quasiconcave and differentiable on $\R^n_{++}$ with strictly positive partial derivatives there, as assumed by Theorem 2.3. 

In exam for this course, again, if the function is one- dimension, you should prove it; if it's a higher-dimension function, the proof is not required.
\end{mdframed}

\begin{mdframed}[backgroundcolor=blue!20,linecolor=white]
Like Jehle \& Reny 1.51, you can actually transform $v(p_1,p_2,1)$  into a function of only $p_1$ or $p_1$ using $p_1x_1 + p_2x_2 = 1$.

$$p_1 = \frac{1 - p_2x_2}{x_1}$$

Substitute into  $v(p_1,p_2,1)$ to have:
$$v(p_1,p_2,1) = [\frac{1 - p_2x_2}{x_1}]^\alpha p_2^\beta$$

Since the question ask you to minimize $v(p_1,p_2,1)$, if you solve $\frac{d e(p_2)}{d p_2} = 0$ and get only one solution, it is the solution.

\begin{align*}
\frac{d e(p_2)}{d p_2} =  \alpha(\frac{1-p_2x_2}{x_1})^{\alpha -1}(\frac{-x_2}{x_1})p_2^\beta +  \frac{1 - p_2x_2}{x_1}^\alpha \beta p_2^{\beta -1} &= 0 \\
 \alpha(\frac{1-p_2x_2}{x_1})^{\alpha -1}(\frac{x_2}{x_1})p_2^\beta &= \frac{1 - p_2x_2}{x_1}^\alpha \beta p_2^{\beta -1} \\
 \alpha(\frac{x_1}{1-p_2x_2})(\frac{x_2}{x_1})p_2 &= \beta  \\
\alpha(\frac{x_2}{1-p_2x_2}) p_2 &= \beta  \\
\alpha x_2 p_2 &= \beta-\beta x_2 p_2  \\
(\alpha x_2 + \beta x_2) p_2&= \beta  \\
 p^*_2&= \frac{\beta}{(\alpha + \beta)x_2} 
\end{align*}
You then solve $p^*_1$ with the budget constraint
\end{mdframed}






%%%%%%%%%%%%%%%%%%%%%%%%%%%%%%%%%%%%%%%%%%%%%%%%%%%%%%%%%%%%%%%%%%%%%%%%%%%%%%%%%%%%%%%%%%%%%%
\section{Jehle \& Reny 2.5(a)}
Consider the solution, $e(p, u) = up_1^{\alpha_1}p_2^{\alpha_2}p_3^{\alpha_3}$ at the end of Example 2.3.
Derive the indirect utility function through the relation $e(p, v(p, y)) = y$ and verify Roy's
identity.

%%%%%%%%%%%%%%%%%%%%%%%%%%%%%%%%%%%%%%%%%%%%%%%%%%%%%%%%%%%%%%%%%%%%%%%%%%%%%%%%%%%%%%%%%%%%%%
\section{Jehle \& Reny 2.7}
Derive the consumer's \textbf{inverse} demand functions, $p_1(x_1, x_2)$ and $p_2(x_1, x_2)$, 
when the utility function is of the Cobb-Douglas form, $u(x_1, x_2) = Ax_1^{\alpha}x_2^{1-\alpha}$
for $0 < \alpha < 1$.

\end{document}
